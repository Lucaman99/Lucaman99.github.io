\documentclass[10pt, oneside]{article} 
\usepackage{amsmath, amsthm, amssymb, calrsfs, wasysym, verbatim, bbm, color, graphics, geometry, hyperref, biblatex, mathtools}
\usepackage[framemethod=TikZ]{mdframed}
\usepackage{tcolorbox}

\hypersetup{
	colorlinks=true,
	linkcolor=blue,
	urlcolor=blue
}

\addbibresource{ref.bib}

\geometry{tmargin=.75in, bmargin=.75in, lmargin=.75in, rmargin = .75in}
\setlength\parindent{0pt}

\tcbuselibrary{theorems}
\newtcbtheorem
  []% init options
  {problem}% name
  {Problem}% title
  {%
    fonttitle=\bfseries,
  }% options
  {prob}% prefix

\newcommand{\R}{\mathbb{R}}
\newcommand{\C}{\mathbb{C}}
\newcommand{\Z}{\mathbb{Z}}
\newcommand{\N}{\mathbb{N}}
\newcommand{\Q}{\mathbb{Q}}
\newcommand{\Cdot}{\boldsymbol{\cdot}}

\newtheorem{thm}{Theorem}
\newtheorem{defn}{Definition}
\newtheorem{conv}{Convention}
\newtheorem{rem}{Remark}
\newtheorem{lem}{Lemma}
\newtheorem{cor}{Corollary}
\newtheorem{prop}{Proposition}

\newcommand{\tr}{\mathrm{Tr}}


\title{Spivak Problem Set 0.1}
\author{Jack Ceroni \thanks{jackceroni@gmail.com}}
\date{September 2020}

\begin{document}

\maketitle
\tableofcontents

\vspace{.25in}


\textit{Quick disclaimer: I am using the third edition of Spivak (not the fourth), so there is a non-zero probability that the
	question assigned by Professor Meinrenken are different than the ones solved here (due to question numbering changing
  between editions).}

\section{Chapter 1}

\subsection{Problem 3}

\begin{enumerate}

\item $\frac{a}{b} \ = \ $
  \item $\frac{a}{b} \ + \ \frac{c}{d} \ = \ \frac{ad}{bd} \ + \ \frac{cb}{db} \ = \ (db)^{-1} (ad \ + \ cb) \ = \ \frac{ad \ + \ cb}{db}$
	\item $(ab)^{-1} \ = \ e \cdot e \cdot (ab)^{-1} \ = \ (a a^{-1}) \cdot (b b^{-1}) \cdot (ab)^{-1} \ = \
		a \cdot a^{-1} \cdot b \cdot b^{-1} \cdot (ab)^{-1} \ = \ (a^{-1} b^{-1}) \cdot (ab) \cdot (ab)^{-1} \ = \
		a^{-1} b^{-1}$
	\item $\frac{a}{b} \cdot \frac{c}{d} \ = \ (a b^{-1}) \cdot (c d^{-1}) \ = \ a \cdot c \cdot d^{-1} \cdot b^{-1}
		\ = \ (ac) \cdot (db)^{-1} \ = \ \frac{ac}{db}$
	\item $\frac{a / b}{c / d} \ = \ \frac{a}{b} \cdot \Big( \frac{c}{d} \Big)^{-1} \ = \ \frac{a}{b} \cdot (c d^{-1})^{-1}
		\ = \ \frac{a}{b} \cdot \big( c^{-1} (d^{-1})^{-1} \big) \ = \ \frac{a}{b} \cdot
		(c^{-1} \cdot (d^{-1})^{-1} \cdot e) \ = \ \frac{a}{c} \cdot
		(c^{-1} \cdot (d^{-1})^{-1} \cdot (d d^{-1})) \ = \ \frac{a}{b} \cdot (c^{-1} d) \ = \
		\frac{a}{b} \cdot \frac{d}{c} \ = \
		\frac{ad}{bc}$

\end{enumerate}

\subsection{Problem 8}

\begin{lem}
  If we assume that modified definition of an order relation (see Problem 8 in Spivak), then the Trichotomy law holds.
\end{lem}

\begin{proof}
  Consider the pair of numbers $0$ and $a$. By the first axiom, it follows that either $a \ = \ 0$, $a \ > \ 0$ (which implies that $a \ \in \ P$, by definition of $P$), or,
  $a \ < \ 0$. Using the third axiom, we add $-a$ to both sides to get: $a \ + \ (-a) \ = \ 0 \ < \ -a$. Thus, in this case, it follows that $-a \ \in \ P$. This proves the Trichotomy law.
\end{proof}

\begin{lem}
  The modified definition of an order relation implies that if $a, \ b \ \in \ P$, then $a \ + \ b \ \in \ P$.
\end{lem}

\begin{proof}
  Assume that $a, \ b \in \ P$. It follows that $0 \ < \ a$ and $0 \ < \ b$. By the third axiom, we have: $0 \ + \ a \ = \ a \ < \ b \ + \ a$. Thus, we have:
  \newpage
  $$0 \ < \ a \ < \ b \ + \ a$$

  By the second axiom, we have $0 \ < \ a \ + \ b$. Thus, $a \ + \ b \ \in \ P$.
\end{proof}

\begin{lem}
  The modified definition of an order relation implies that if $a, \ b \ \in \ P$, then $a \cdot b \ \in \ P$.
\end{lem}

\begin{proof}
  Assume that $a, \ b \ \in \ P$. We thus have $0 \ < \ a$ and $0 \ < \ b$. It follows from the fourth axiom that $0 \cdot b \ = \ 0 \ < \ a \cdot b$. Thus,
  $a \cdot b \ \in \ P$.
\end{proof}

\subsection{Problem 13}

\begin{prop}
  For two numbers $x$ and $y$, we have:

  $$\max (x,  y) \ = \ \frac{x \ + \ y \ + \ |y \ - \ x|}{2}$$

  $$\min (x,  y) \ = \ \frac{x \ + \ y \ - \ |y \ - \ x|}{2}$$
  \end{prop}

\begin{proof}
Let us pick two numbers, $x$ and $y$. Assume, without loss of generality, that $x \ < \ y$. Consider the quantity:

$$\text{max} (x,  y) \ = \ \frac{x \ + \ y \ + \ |y \ - \ x|}{2}$$

Since $x \ < \ y$, it follows that $0 \ < \ y \ - \ x$, and we have $|y \ - \ x| \ = \ y \ - \ x$:

$$\text{max} (x,  y) \ = \ \frac{x \ + \ y \ + \ y \ - \ x}{2} \ = \ \frac{2y}{2} \ = \ y$$

which is in fact the larger of the pair $x, \ y$. We can also consider the quantity:

$$\text{min} (x,  y) \ = \ \frac{x \ + \ y \ - \ |y \ - \ x|}{2} \ = \ \frac{x \ + \ y \ - \  y \ + \ x}{2} \ = \ x$$

which is in fact the smaller of the pair $x, \ y$. Thus, the proof is complete.
\end{proof}

\newline

\texit{Derive expressions for $\max(x, \ y, \ z)$ and $\min(x, \ y, \ z)$.}

Since $\max(x, \ y, \ z) \ = \ \max(x, \ \max(y, \ z))$, we have:

$$\max(x, \ y, \ z) \ = \ = \ \frac{x \ + \ \max(y, \ z) \ + \ |\max(y, \ z) \ - \ x|}{2}$$

\section{Chapter 28}

\subsection{Problem 5}

\begin{lem}
	For any field, we have:

	$$\underbrace{(e \ + \ \cdots \ + \ e)}_{m \ \text{times}} \ \cdot \ \underbrace{(e \ + \ 
	\cdots \ + \ e)}_{n \ \text{times}} \ = \ \underbrace{(e \ + \ \cdots \ + \ e)}_{mn \ \text{times}}$$

	for all natural numbers $n$ and $m$.
\end{lem}

\begin{proof}
	Pick ome arbitrary natural number $m$. We proceed by induction. Clearly, the lemma is true in 
	the case of $n \ = \ 1$. Let us assume the case of $n$. Consider the case of $n \ + \ 1$. We have:
	
	$$\underbrace{(e \ + \ \cdots \ + \ e)}_{m \ \text{times}} \ \cdot \ \underbrace{(e \ + \ 
	\cdots \ + \ e)}_{n \ + \ 1 \ \text{times}} \ = \ \underbrace{(e \ + \ \cdots \ + \ e)}_{m \ \text{times}} \ \cdot \ \big[ \underbrace{(e \ + \ 
\cdots \ + \ e)}_{n \ \text{times}} \ + \ e \big]$$

  Now, we use the distributive property of fields and the definition of the identity, along with the assumtpion that
  the lemma holds true in the case of $n$ to get:

$$\Rightarrow \ \big[ \underbrace{(e \ + \ 
\cdots \ + \ e)}_{mn \ \text{times}} \ + \ e \ \cdot \  \underbrace{(e \ + \ \cdots \ + \ e)}_{m \ \text{times}} \big] \ = \ 
\big[ \underbrace{(e \ + \ 
\cdots \ + \ e)}_{mn \ \text{times}} \ + \ \underbrace{(e \ + \ \cdots \ + \ e)}_{m \ \text{times}} \big] \ = \ 
\underbrace{(e \ + \ 
\cdots \ + \ e)}_{m(n \ + \ 1) \ \text{times}}$$ 

So the lemma is proved.

\end{proof}

\begin{thm}
	If in some field $F$ we have:

	$$\underbrace{e \ + \ \cdots \ + \ e}_{n \ \text{times}} \ = \ 0$$

	then the smallest $n$ for which this is true is prime.
\end{thm}

\begin{proof}
	Assume that $n$ isn't prime. It follows that we can write $n$ as a product of at least two whole numbers 
	less than $n$ and greater than $1$. Thus, $n \ = \ ab$. By the previous lemma, we have:
\newpage	
	$$\underbrace{e \ + \ \cdots \ + \ e}_{n \ \text{times}} \ = \ \underbrace{(e \ + \ \cdots \ + \ e)}_{a \ \text{times}} \ \cdot \ \underbrace{(e \ + \ 
	  \cdots \ + \ e)}_{b \ \text{times}} \ = \ 0$$

  In a field, we know that $a \ \cdot \ 0 \ = \ a$, as $0$ is the element of the field such that $a \ + \ 0 \ = \ a$. We then have (by distribution) that
  $(a \cdot a) \ + \ (a \cdot 0) \ = \ (a \cdot a) \ = \ (a \cdot a) \ + \ 0$. By left cancellation, we have $a \cdot 0 \ = \ 0$. Assume that both the right-hand sums
  of $e$ (for $a$ and $b$) are non-zero. It follows that they have inverses. Let us denote the two sums by $A$ and $B$. It follows that:

  $$e \ = \ A^{-1} A B^{-1} B \ = \ (A^{-1} B^{-1}) \cdot (A B) \ = \ (A^{-1} B^{-1}) \cdot 0 \ = \ 0$$

  which is a contradiction to the definition of a field, as the additive and multiplicative identities must be different. Thus, at least one of these sums is equal to $0$ it follows that either $a$ or $b$ is a whole number
  less than $n$ such that:

  $$\underbrace{e \ + \ \cdots \ + \ e}_{a \ \text{or} \ b \ \text{times}} \ = \ 0$$

  which is a contradiction. Thus, $n$ must be prime.

\end{proof}

\subsection{Problem 6}

\begin{lem}
  For some field $F$ with a finite number of elements, there exist distinct natural numbers $m$ and $n$ such that:

  $$\underbrace{e \ + \ \cdots \ + \ e}_{m \ \text{times}} \ = \ \underbrace{e \ + \ \cdots \ + \ e}_{n \ \text{times}}$$
\end{lem}

\begin{proof}
  Let $|F| \ = \ k$ be the cardinality of the set defining the field (which we know is some finite natural number, $k$). Let:

  $$E(n) \ = \ \underbrace{e \ + \ \cdots \ + \ e}_{n \ \text{times}}$$

  Now, consider the set $\{E(1), \ E(2), \ ..., \ E(k), \ E(k \ + \ 1)\}$. It follows that there must exist two elements of
  this set that are equal, or else we would have a subset of $F$ that contains $k \ + \ 1$ \textbf{distinct} elements, a clear contradiction.
  Hence, there exist $m$ and $n$ such that:

  $$\underbrace{e \ + \ \cdots \ + \ e}_{m \ \text{times}} \ = \ \underbrace{e \ + \ \cdots \ + \ e}_{n \ \text{times}}$$

\end{proof}

\begin{thm}
  In a field $F$ with a finite number of elements, there exists some natural number $r$ such that:

  $$\underbrace{e \ + \ \cdots \ + \ e}_{r \ \text{times}} \ = \ 0$$
\end{thm}

\begin{proof}
  By the previous lemma, we know there exist $m$ and $n$ such that $E(m) \ = \ E(n)$. Without loss of generality, let $n \ < \ m$ (the two numbers
  are distinct, so one is larger than the other). We have:

  $$0 \ + \ \underbrace{e \ + \ \cdots \ + \ e}_{n \ \text{times}} \ = \ \underbrace{e \ + \ \cdots \ + \ e}_{m \ \text{times}} \ = \ \underbrace{e \ + \ \cdots \ + \ e}_{m \ - \ n \ \text{times}} \ + \
  \underbrace{e \ + \ \cdots \ + \ e}_{n \ \text{times}}$$

  So by right cancellation, we have:

  $$\underbrace{e \ + \ \cdots \ + \ e}_{m \ - \ n \ \text{times}} \ = \ 0$$

  It follows that $r \ = \ m \ - \ n$ and the theorem is proved.
  \end{proof}

\end{document}
