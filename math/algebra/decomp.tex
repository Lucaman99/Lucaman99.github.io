\documentclass[10pt, oneside]{article} 
\usepackage{amsmath, amsthm, amssymb, calrsfs, wasysym, verbatim, bbm, color, graphics, geometry, hyperref, biblatex, mathtools}
\usepackage[framemethod=TikZ]{mdframed}
\usepackage{tcolorbox}

\hypersetup{
	colorlinks=true,
	linkcolor=blue,
	urlcolor=blue
}

\addbibresource{ref.bib}

\geometry{tmargin=.75in, bmargin=.75in, lmargin=1.25in, rmargin=1.25in}
\setlength\parindent{0pt}

\tcbuselibrary{theorems}
\newtcbtheorem
    []% init options
    {problem}% name
    {Problem}% title
    {%
      fonttitle=\bfseries,
    }% options
    {prob}% prefix

    \newcommand{\R}{\mathbb{R}}
    \newcommand{\C}{\mathbb{C}}
    \newcommand{\Z}{\mathbb{Z}}
    \newcommand{\N}{\mathbb{N}}
    \newcommand{\Q}{\mathbb{Q}}
    \newcommand{\Cdot}{\boldsymbol{\cdot}}

    \newtheorem{thm}{Theorem}
    \newtheorem{defn}{Definition}
    \newtheorem{conv}{Convention}
    \newtheorem{rem}{Remark}
    \newtheorem{lem}{Lemma}
    \newtheorem{cor}{Corollary}
    \newtheorem{prop}{Proposition}

    \newcommand{\tr}{\mathrm{Tr}}


    \title{The Decomposition Theorem Done Wrong}
    \author{Jack Ceroni}
    \date{December 2020}

    \begin{document}

    \maketitle
    \tableofcontents

    \vspace{.25in}

    \newpage

   \section{Introduction}

    Marco's presentation of the decomposition theorem could not have been better. However, it's a lot of stuff to remember, so I thought I would try to write it up and
    explain it in a somewhat intuitive way here.
    \newline

    If you notice any mistakes in these notes, please do not hesitate to send me an email or
    send me a message on the Math Physics Specialist Discord server.

   \section{The Decomposition Theorem}



    \end{document}
