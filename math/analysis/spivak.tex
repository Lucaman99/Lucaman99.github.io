\documentclass[10pt, oneside]{article} 
\usepackage{amsmath, amsthm, amssymb, calrsfs, wasysym, verbatim,
  bbm, color, graphics, geometry, hyperref, biblatex, mathtools}
\usepackage[framemethod=TikZ]{mdframed}
\usepackage{tcolorbox}

\hypersetup{
	colorlinks=true,
	linkcolor=blue,
	urlcolor=blue
}

\addbibresource{ref.bib}

\geometry{tmargin=.75in, bmargin=.75in, lmargin=.75in, rmargin = .75in}
\setlength\parindent{0pt}

\newenvironment{problem}[2][Problem]{\begin{trivlist}
\item[\hskip \labelsep {\bfseries #1}\hskip \labelsep {\bfseries #2.}]}{\end{trivlist}}

\tcbuselibrary{theorems}

    \newcommand{\R}{\mathbb{R}}
    \newcommand{\C}{\mathbb{C}}
    \newcommand{\Z}{\mathbb{Z}}
    \newcommand{\N}{\mathbb{N}}
    \newcommand{\Q}{\mathbb{Q}}
    \newcommand{\Cdot}{\boldsymbol{\cdot}}

    \newtheorem{thm}{Theorem}
    \newtheorem{defn}{Definition}
    \newtheorem{conv}{Convention}
    \newtheorem{rem}{Remark}
    \newtheorem{lem}{Lemma}
    \newtheorem{cor}{Corollary}
    \newtheorem{prop}{Proposition}

    \newcommand{\tr}{\mathrm{Tr}}


    \title{Spivak Notes, Problems, and Solutions}
    \author{Jack Ceroni}
    \date{October 2020}

    \begin{document}

    \maketitle
    \tableofcontents

    \vspace{.25in}

    \section{Introduction}

    The goal of this set of notes is to solve the most challenging problems in Spivak, and
    write up the solutions in a clean and concise way. I apologize in advance for any
    possible mistakes, or instances in which I may skip over certain important points.

    \newpage

    \section{Chapter 3}

    \begin{problem}{3.17}
      Prove that if $f(x + y) = f(x) + f(y)$ and $f(x \cdot y) = f(x) \cdot f(y)$, where $f(x) \neq 0$, then $f(x) = x$ for all $x$.
    \end{problem}

    \begin{proof}
      We go through the steps of the proof, as organized in Spivak:

      \begin{enumerate}
      \item Clearly, we will have $f(1) = f(1 \cdot 1) = f(1) \cdot f(1)$, so either $f(1) = 0$ or $f(1) = 1$. If we assume that $f(1) = 0$,
        then this would imply that $f(n) = 0$ for all $n$ (we can prove this by induction, assuming that $f(n) = 0$, and noting that $f(n + 1) = f(n) + f(1) = 0$).
        This is a contradiction to our initial assumption, so $f(1) = 1$.

      \item First, we note that:

        $$f(0) = f(0 + 0) = f(0) + f(0) \ \Rightarrow \ f(0) = 0$$

        Next, we note that $f(n) = n$, for natural $n$. We prove this by induction, first assuming that $f(n) = n$, then noting that $f(n + 1) = f(n) + f(1) = n + 1$.
        We then note that $f(-n) = n - n + f(-n) = -n + f(n) + f(-n) = -n + f(0) = -n$. Thus, $f$ is the identity for all integers.

        Now, we can see that:

        $$f\Big(\frac{1}{b}\Big) = \frac{b}{b} \cdot f\Big(\frac{1}{b}\Big) = \frac{1}{b} \cdot f(b) \cdot f\Big(\frac{1}{b}\Big) = \frac{1}{b} \cdot f(1) = \frac{1}{b}$$

        Thus,

        $$f\Big(\frac{a}{b}\Big) = f(a) \cdot f\Big( \frac{1}{b} \Big) = \frac{a}{b}$$

      \item Assume that $x > 0$. It then follows that $\sqrt{x}$ is well-defined and greater than $0$. We then have:

        $$f(x) = f(\sqrt{x} \cdot \sqrt{x}) = f(\sqrt{x}) f(\sqrt{x}) = f(\sqrt{x})^2$$

        we know that for any real number $r$, we have $r^2 \geq 0$, so $f(x) \geq 0$. Assume that $f(x) = 0$.
        Since $x > 0$, this would imply that:

        $$f(1) = f\Big( \frac{x}{x} \Big) = f(x) \cdot f\Big( \frac{1}{x} \Big) = 0$$

        a clear contradiction. Thus, $f(x) > 0$.

      \item If $x > y$, then we know that $x - y > 0$, so it follows from previous result that:

        $$f(x - y) > 0 \ \Rightarrow \ f(x) - f(y) > 0 \ \Rightarrow \ f(x) > f(y)$$

      \item Assume that there exists some $x$ such that $x < f(x)$. Since there exists a rational number between any two reals, it follows that we have:

        $$x < \frac{a}{b} < f(x)$$

        for some $a/b$. From the previous result, we then get $f(x) < f(a/b)$, a clear contradiction to the right-most inequality above. Similarly, if we assume
        that $f(x) < x$, we will have:

        $$f(x) < \frac{a}{b} < x$$

        so $f(a/b) < f(x)$, another contradiction. It follows that $f(x) = x$, and we have proved the proposition.

        \end{enumerate}
    \end{proof}

    \begin{problem}{3.20B}
      If a function satisfies:

      $$f(y) - f(x) \leq (x - y)^2$$

      for all $, y \in \mathbb{R}$, then $f(x) = c$ for some $c$ and all $x$
    \end{problem}

    \textit{Part B is the interesting part of this problem, so I skipped writing out Part A}

    \begin{proof}
      Assume that there exist distinct $x$ and $y$ such that $f(x) \neq f(y)$. Without loss of generality, let $f(x) < f(y)$. It follows that:

      $$f(y) - f(x) \leq (y - x)^2$$

      Consider what happens when we split up the interval from $x$ to $y$ into $n$ ``chunks''. We let:

      $$z_j = \Big(1 - \frac{j}{n} \Big) x + \frac{j}{n} y$$

      so we get $z_0 = x$ and $z_n = y$. Clearly the distance between $z_i$ and $z_{j - 1}$ is given by:

      $$z_{j} - z_{j - 1} = \Big(1 - \frac{j}{n} \Big) x + \frac{j}{n} y - \Big(1 - \frac{j - 1}{n} \Big) x - \frac{j - 1}{n} y = \frac{y - x}{n}$$

      It follows that:

      $$f(z_{j}) - f(z_{j - 1}) \leq (z_{j} - z_{j - 1})^2 =  \frac{(y - x)^2}{n^2}$$

      Now comes the crucial step. Notice that

      $$\displaystyle\sum_{j = 1}^{n} \big( f(z_{j}) - f(z_{j - 1}) \big) = f(z_{n}) - f(z_{0}) = f(y) - f(x)$$

      as the rest of the terms cancel. Thus, we will have:

      $$\displaystyle\sum_{j = 1}^{n} \big( z_{j} - z_{j - 1} \big) \leq \displaystyle\sum_{j = 1}^{n}
      \frac{(y - x)^2}{n^2} \ \Rightarrow \  f(y) - f(x) \leq n \cdot \frac{(y - x)^2}{n^2} = \frac{(y - x)^2}{n}$$

      for all possible values of $n$. Since we have assume $f(y) \neq f(x)$, it follows that $f(y) - f(x)$ is a positive real number, and that:

      $$\epsilon = \frac{f(y) - f(x)}{(y - x)^2}$$

      is a positive real as well. Thus, this implies that there exists a real number $\epsilon$ such that for any positive integer $n$:

      $$\epsilon \leq \frac{1}{n}$$

      But this clearly contradicts the Archimedean property of the real numbers. We have derived a contradiction, so it follows that for any $x$ and $y$, $f(x) = f(y)$. Thus,
      the function $f$ is constant.

    \end{proof}

    \section{Chpater 5}

    \begin{lem}[Uniqueness of Limits]
      The limit of a function is unique: If a function $f$ approaches $\ell_1$ as $x$ approaches $a$, and $f$ approaches $\ell_2$ as
      $x$ approaches $a$, then $\ell_1 = \ell_2$.
    \end{lem}

    \begin{proof}
      Suppose the the function $f$ approaches $\ell_1$ and $\ell_2$. It follows that given some $\epsilon > 0$, we can choose $\delta_1$ and $\delta_2$ such that:

      $$|x - a| < \delta_1 \ \Rightarrow \ |f(x) - \ell_1| < \epsilon$$
      $$|x - a| < \delta_2 \ \Rightarrow \ |f(x) - \ell_2| < \epsilon$$

      Assume that $\ell_1 \neq \ell_2$, so $|\ell_1 - \ell_2| > 0$. Let us then pick $\epsilon = \frac{|\ell_1 - \ell_2|}{2}$. We can then pick $\delta_1$ and $\delta_2$ corresponding
      to this $\epsilon$. We then let $\delta = \text{min}(\delta_1, \ \delta_2)$ so:

      $$|x - a| < \delta \ \Rightarrow \ |f(x) - \ell_1| < \ \epsilon \ \ \ \ \ \text{and} \ \ \ \ \ |f(x) - \ell_2| < \epsilon$$

      It then follows that:

      $$|x - a| < \delta \ \Rightarrow \ |f(x) - \ell_1| + |f(x) - \ell_2| < 2\epsilon = |\ell_1 - \ell_2|$$

      We know that there exists some $x_0$ such that $|x_0 - a| < \delta$, which implies that:

      $$|\ell_1 - \ell_2| \leq |f(x_0) - \ell_1| + |f(x_0) - \ell_2| < |\ell_1 - \ell_2|$$

      a clear contradiction. It follows that $\ell_1$ must equal $\ell_2$.

    \end{proof}

    \begin{lem}[Sums of Limits]
      If $\lim_{x \to a} f(x) = m$ and $\lim_{x \to a} g(x) = \ell$, then $\lim_{x \to a} (f + g)(x) = m + \ell$.
    \end{lem}

    \begin{proof}

      Let us pick some $\epsilon > 0$. We will have, for $\epsilon/2$:

      $$|x - a| < \delta_1 \ \Rightarrow \ |f(x) - m| < \epsilon/2$$
      $$|x - a| < \delta_2 \ \Rightarrow \ |g(x) - \ell| < \epsilon/2$$

      we choose $\delta = \text{min}(\delta_1, \ \delta_2)$, giving us:

      $$|x - a| < \delta \ \Rightarrow \ |f(x) - m| + |g(x) - \ell| < \epsilon$$

      Then, given $x$ such that $|x - a| < \delta$, we have:

      $$|f(x) + g(x) - (m + \ell)| \leq |f(x) - m| + |g(x) - \ell| < \epsilon$$

      Thus, given $\epsilon$, we can choose a $\delta$. It follows by definition that $\lim_{x \to a} (f + g)(x) = m + \ell$.

    \end{proof}

    \begin{problem}{5.20}
      If $f(x) = x$ for rational $x$ and $f(x) = -x$ for irrational $x$, show that $\lim_{x \to a} f(x)$ does not exist for $a \neq 0$.
    \end{problem}

    \begin{proof}
      Assume that there exists some non-zero $a$ such that:

      $$\lim_{x \to a} f(x) = L$$

      It follows that for any $\epsilon > 0$, we can choose a $\delta$ such that if $|x - a| < \delta$, then $|f(x) - L| < \epsilon$. We begin by considering the case when $a > 0$.
      We let $\epsilon = a$ and assume that we can choose a $\delta$ such that:

      $$|x - a| < \delta \ \Rightarrow \ |f(x) - L| < a$$

      Now, since there exists a rational and an irrational number between any two reals, we pick rational $r$ and irrational $i$ from the interval $(a, \ a + \delta)$. We will then have:

      $$|r - a| < \delta \ \Rightarrow \ |r - L| < a$$
      $$|i - a| < \delta \ \Rightarrow \ |-i - L| = |i + L| < a$$

      So we will have:

      $$|(r - L) + (i + L)| = |r + i| \leq |r - L| + |i + L| < 2a$$

      But this is a contradiction, as $a < i, \ r$, so $2a < i + r$. Thus, we can choose no such $\delta > 0$, and the limit does not exist.
      \newline

      Now, assume that $a < 0$. We let $\epsilon = |a|$ and assume that we can choose a corresponding $\delta$. We then choose rational and irrational $r, i \in (a - \delta, a)$.
      Similar to above, we get:

      $$|(r - L) + (i + L)| = |r + i| \leq |r - L| + |i + L| < 2|a|$$

      But this is a contradiction, as $i, r < a$, so $i + r < 2a$, which implies that $|i + r| > 2|a|$ (as $a$, $i$, and $r$ are negative). Thus, we can choose no such $\delta$, and the limit does
      not exist.
      \newline

      We conclude that the limit does not exist for any $a > 0$, and any $a < 0$, making $a = 0$ the only point at which the limit exists.
    \end{proof}

    \begin{problem}{5.12}

      Suppose that $f(x) \leq g(x)$ for all $x$. Prove that $\lim_{x \to a} f(x) \leq \lim_{x \to a} g(x)$, assuming the limits exist.

    \end{problem}

    \begin{proof}

      We let the first limit be denoted by $\ell_f$ and the second by $\ell_g$. Assume that $\ell_f > \ell_g$. It follows that
      $\ell_f - \ell_g$ is a positive real number, so we let $\epsilon = (\ell_f - \ell_g)/2$. Now, by definition of the limits, we
      can choose $\delta_1$ and $\delta_2$ such that:

      $$0 < |x - a| < \delta_1 \ \Rightarrow \ |f(x) - \ell_f| < (\ell_f - \ell_g)/2$$
      $$0 < |x - a| < \delta_2 \ \Rightarrow \ |g(x) - \ell_g| < (\ell_f - \ell_g)/2$$.

      We let $\delta = \min\{\delta_1, \ \delta_2\}$. We then have:

      $$0 < |x - a| < \delta \ \Rightarrow \ |\ell_f - f(x) + g(x) - \ell_g| \leq |f(x) - \ell_f| + |g(x) - \ell_g| < \ell_f - \ell_g$$

      So we have:

      $$|(g(x) - f(x)) + (\ell_f - \ell_g)| < \ell_f - \ell_g$$

      but since $f(x) \leq g(x)$ and $\ell_g < \ell_f$, both numbers in the brackets will be greater than or equal to $0$, so:

      $$g(x) - f(x) + \ell_f - \ell_g < \ell_f - \ell_g$$

      which is a contradiction. Thus, $\ell_f \leq \ell_g$ and the proof is complete.

      \end{proof}

    \begin{problem}{5.23}

      Let $f$ be a function with the following property: if $g$ is a function for which $\lim_{x \to 0} g(x)$ does not exists, then $\lim_{x \to 0} [f(x) \cdot g(x)]$ also does not exist. Prove
      that $f$ has this property if and only if $\lim_{x \to 0} f(x)$ exists.

    \end{problem}

    \begin{proof}

      We start by considering the case where $\lim_{x \to 0} f(x)$ exists and is equal to $m \neq 0$. Let $g(x)$ be a function such that $\lim_{x \to 0} g(x)$ does not exist. Assume that $\lim_{x \to 0} [f(x) \cdot g(x)]$
      exists, so it is equal to some real $\ell$. We then have:

     $$\frac{\lim_{x \to 0} [f(x) \cdot g(x)]}{\lim_{x \to 0} f(x)} = \lim_{x \to 0} \Big(\frac{f(x) \cdot g(x)}{f(x)}$$

    \end{proof}

    \begin{problem}{5.24}

      Suppose that $A_n$ is, for each natural $n$, some finite set of numbers of $[0, \ 1]$, and
      that $A_n$ and $A_m$ are disjoint if $n \neq m$. Define $f$ as follows:

      $$f = \begin{cases}
        1/n & x \in A_n \\
        0 & x \notin A_n \forall n \in \mathbb{N}
      \end{cases}
      $$

      Prove that $\lim_{x \to a} f(x) = 0$ for any $a \in [0, \ 1]$.

    \end{problem}

    \begin{proof}

      Let us pick some $a \in [0, \ 1]$ and some $\epsilon > 0$. By the Archimedean property, we can pick some natural $n$ such that
      $1/n < \epsilon$. Since each $A_m$ contains only a finite number of elements, it follows that the union of the collection
      of set $\{A_{1}, \ ..., \ A_{n - 1}\}$ also contains a finite number of elements.
      \newline

      By definition of $f$, this implies that there are a finite
      number of $x \in [0, \ 1]$ such that $1/n < f(x)$. We denote the set of such $x$ by $X$. Then, we let:

      $$\delta = \min\{|x - a| \ | \ x \in X - \{a\}\}$$

      where the minimum of the set is well-defined, as $X$ contains a finite number of elements. It then follows that
      given some $y$ such that $0 < |y - a| < \delta$, $y$ cannot possibly be in $X$, so it must be true that $f(x) \leq 1/n < \epsilon$.

      \end{proof}

    \end{document}
