\documentclass[10pt, oneside]{article} 
\usepackage{amsmath, amsthm, amssymb, wasysym, verbatim, bbm, color, graphics, geometry, hyperref, biblatex}
\usepackage[framemethod=TikZ]{mdframed}
\usepackage{tcolorbox}

\hypersetup{
	colorlinks=true,
	linkcolor=blue,
	urlcolor=blue
}

\addbibresource{ref.bib}

\geometry{tmargin=.75in, bmargin=.75in, lmargin=1.25in, rmargin = 1.25in}
\setlength\parindent{0pt}

\newenvironment{problem}[2][Problem]{\begin{trivlist}
\item[\hskip \labelsep {\bfseries #1}\hskip \labelsep {\bfseries #2.}]}{\end{trivlist}}

\tcbuselibrary{theorems}

    \newcommand{\R}{\mathbb{R}}
    \newcommand{\C}{\mathbb{C}}
    \newcommand{\Z}{\mathbb{Z}}
    \newcommand{\N}{\mathbb{N}}
    \newcommand{\Q}{\mathbb{Q}}
    \newcommand{\Cdot}{\boldsymbol{\cdot}}

    \newtheorem{thm}{Theorem}
    \newtheorem{defn}{Definition}
    \newtheorem{conv}{Convention}
    \newtheorem{rem}{Remark}
    \newtheorem{lem}{Lemma}
    \newtheorem{cor}{Corollary}
    \newtheorem{prop}{Proposition}

    \newcommand{\tr}{\mathrm{Tr}}
    \newcommand{\pow}{\mathcal{P}}
     \newcommand{\Null}{\text{null}}


    \title{Artin Reading Group Notes}
    \author{Jack Ceroni}
    \date{December 2020}

    \begin{document}


    \maketitle
    \tableofcontents

    \vspace{.25in}

    \newpage

    \section{Introduction}

    The following are notes and solutions for Week 1 of the Artin reading group.

    \section{Chapter 2: Basic Group Theory}

    \subsection{Recap of Lagrange's Theorem and Related Results}

    \begin{defn}
      We define the left coset of a subgroup $H$ of some group $G$ to be the set:

      $$aH = \{ ah \ | \ h \in H\}$$
    \end{defn}

    Note that the collection of cosests forms a partition of $G$.

    \begin{prop}
      There exists a bijection from $H$ to $aH$.
    \end{prop}

    \begin{proof}
      We define the map $\phi$ such that $\phi(h) = ah$. Clearly, such a map is surjective. In addition, we note that
      if $ah = ah'$, then $h = h'$, which implies that $\phi$ is injective.
      \newline

      Therefore, $\phi$ is a bijection, and $H$ and $aH$ have the same number of elements.
    \end{proof}

    We define $[G : H]$ to be the number of left subgroups of $H$. It follows that $|G| = |H| [G : H]$ (from above). This leads to
    Lagrange's theorem.

    \subsection{Problems}

    \begin{problem}{2.21}
      Prove that the set of elements of finite order in an Abelian group is a subgroup.
    \end{problem}

    \begin{proof}
      Given two elements $x$ and $y$ of finite order, we note that there exists $m$ and $n$ such that $x^{m} = 1$ and $y^{n} = 1$. Since the group is Abelian,
      we know that $(xy)^{r} = x^{r} y^{r}$. Therefore:

      $$(xy)^{mn} = x^{mn} y^{mn} = (x^{m})^{n} (y^{n})^{m} = 1^n 1^m = 1$$

      so $xy$ has finite order as well (namely, $mn$). Clearly, $1$ has finite order, so it is in the group. Finally, for some $x$ of finite order $n$, we note that:

      $$1 = 1^{n} = (x x^{-1})^{n} = x^{n} (x^{-1})^n = (x^{-1})^n$$

      so $x^{-1}$ has finite order (namely, $n$), and is therefore in the group as well. It follows that we have a valid subgroup.
    \end{proof}

    \begin{problem}{3.12}
      Let $G$ be a group, and let $\phi : G \ \rightarrow \ G$ be the map $\phi(x) = x^{-1}$. Prove that $\phi$ is bijective. Then, prove that $\phi$ is an automorphism if and only if
      $G$ is Abelian.
    \end{problem}

    \begin{proof}
      Clearly, $\phi$ is surjective: given $x \in G$, we note that $\phi(x^{-1}) = x$. In addition, assume that $y^{-1} = x^{-1}$. It follows that:

      $$x = x y^{-1} y = x x^{-1} y = y$$

      so $\phi$ is injective. Therefore, $\phi$ is bijective.
      \newline

      First, assume that $\phi$ is an automorphism. We then will have:

      $$xy = \phi(x^{-1}) \phi(y^{-1}) = \phi(x^{-1} y^{-1}) = \phi((yx)^{-1}) = yx$$

      so $G$ is Abelian. Now, assume that $G$ is Abelian. We will then have:

      $$\phi(xy) = (xy)^{-1} = y^{-1} x^{-1} = x^{-1} y^{-1} = \phi(x) \phi(y)$$

      so $\phi$ is an automorphism. This completes the proof.
    \end{proof}

    \begin{problem}{4.11}
      Let $G$, $H$ be cyclic groups generated by $x$ and $y$, with orders $m$ and $n$. What condition must be imposed on $m$ and $n$ such that $\phi(x^{i}) = y^i$ is a
      valid homomorphism?
    \end{problem}

    \begin{prop}
      $\phi(x^i) = y^i$ is a homomorphism if and only if $m = zn$, where $z$ is a positive integer.
    \end{prop}

    \begin{proof}
      First, assume that $m = zn$. We note that in $G$:

      $$x^{i} = 1 x^{i} = x^{mj} x^{i} = x^{i + mj}$$

      for all integer $j \geq 0$. Clearly, these are the only elements to which $x^i$ is equal in $G$. Thus, to show that $\phi$ is a valid
      map, we must show that $x^{i}$ and $x^{i + mj}$ are mapped to the same element of $H$.
      \newline

      We have:

      $$\phi(x^i) = y^i$$

      and:

      $$\phi(x^{i + mj}) = y^{i + mj} = y^i y^{mj} = y^i y^{(zj)n} = y^i$$

      so $\phi$ is a valid map. In addition, we note that:

      $$\phi(x^{a} x^{b}) = \phi(x^{a + b}) = y^{a + b} = y^a y^b = \phi(x^a) \phi(x^b)$$

      Thus, $\phi$ is a valid group homomorphism.
      \newline

      Conversely, assume that $\phi$ is a homomorphism. We will have:

      $$\phi(x^{i}) = \phi(x^{i + m}) = \phi(x^i) \phi(x^m) \ \Rightarrow \ \phi(x^m) = y^m = 1$$

      Since $y$ has order $n$, this implies that $m = in$, for some integer $i \geq 0$. This completes the proof.
    \end{proof}

    \begin{problem}{4.23}
      Let $\phi : G \ \rightarrow \ G'$ be a surjective homomorphism, and let $N$ be a normal subgroup of $G$. Prove that $\phi(N)$ is a normal subgroup of $G'$.
    \end{problem}

    \begin{proof}
      Consider some element $\phi(n) \in \phi(N)$. Now, let us pick some other element $b$ in $G'$. Since $\phi$ is surjective, we note that $b = \phi(a)$. We then have:

      $$\phi(a)^{-1} \phi(n) \phi(a) = \phi(a^{-1}) \phi(n) \phi(a) = \phi(a^{-1} n a)$$

      Since $N$ is normal, it follows that $a^{-1} n a \in N$, so $\phi(a^{-1} n a) \in \phi(N)$. Thus, by definition, $\phi(N)$ is normal as well.
    \end{proof}

    \begin{problem}{5.12}
      Prove that the non-empty cosets of the kernel of $\phi$ are the fibres of $\phi$.
    \end{problem}

    \begin{proof}
      Let $S$ be the set of fibres of $\phi$. Consider some arbitrary non-empty fibre of the form $\phi^{-1}(a) \in S$.
      \newline

      Let $x$ be an element of the fibre. Now, consider some other element $y$ of the fibre (not necessarily distinct). We note that $\phi(x) = \phi(y) = a$,
      so it follows that $\phi(y) \phi(x^{-1}) = \phi(y x^{-1}) = 1$. This implies that $y x^{-1}$ is in the kernel, so $y = nx$, for some $n$ in the nullspace.
      \newline

      In addition, clearly, for any element $xn$, we will have:

      $$\phi(xn) = \phi(x) \phi(n) = \phi(x) = a$$

      Thus, by definition, each fibre is a coset of the above form.
      \newline

      \textbf{Note:} One way to prove this would be to use a similar procedure to what was done above, by showing that every coset is a fibre. However, we choose
      a more ``crafty'' method:
      \newline

      Now, consider some coset $xN$. Assume that $xN$ is not a fibre. Since cosets partition the set, and each fibre is a coset, it follows that $xN$ cannot intersect any fibre.
      But $x \in xN$, which would imply that $x$ is not contained in any fibre of $\phi$, a contradiction to the fact that fibres partition the domain.
      \newline

      Therefore, each coset must be a fibre.
      \newline

      It follows that the set of cosets is equal to the set of fibres, and the proof is complete.
    \end{proof}

    \begin{problem}{7.8}
      Prove the Correspondence Theorem.
    \end{problem}

    \begin{proof}
      We define a map that takes each subgroup $H$ containing the kernel of a homomorphism $\phi$ to $\phi(H)$. We show that such a map defines a bijective correspondence
      between all subgropus of this form in $G$, and all subgroups $H'$ of $G'$.
      \newline

      Consider some subgroup $H'$ of $G'$. We define the set $\phi^{-1}(H')$. Clearly, $N \in \phi^{-1}(H')$, as $1 \in H'$.
      \newline

      In addition, we note that given $x, \ y \in \phi^{-1}(H')$,
      we will have $\phi(xy) = \phi(x) \phi(y)$. Clearly, $\phi(x), \ \phi(y) \in H'$, and so too will be $\phi(x) \phi(y)$. Therefore, $xy \in \phi^{-1}(H')$. Similar arguments show that
      $\phi^{-1}(H')$ contains inverses and the identity. Therefore, $\phi^{-1}(H')$ is a subgroup.
      \newline

      It follows that $\phi(\phi^{-1}(H') \subset H'$ (as $\phi$ is surjective), and the map that takes $H$ to $\phi(H)$ is surjective.
      \newline

      Now, consider two subgroups that contain $N$, which we denote $A$ and $B$. Assume that $\phi(A) = \phi(B)$. Consider some $a \in A$. We note that $\phi(a) = \phi(b)$
      for some $b \in B$. But we proved earlier that $a = bn$ for some $n$ in the kernel. We also know that $N \subset B$, so $nb \in B$. This implies that $a \in B$ and $A \subset B$.
      \newline

      Similarly, we can show that $B \subset A$. Therefore, $A = B$. It follows that the map that sends $H$ to $\phi(H)$ is injective, by definition. Now we know that the map is
      surjective and injective, so it is bijective.
      \newline

      Finally, we note that if $H$ is normal, and given some $g \in \phi(H)$, then:

      $$\phi(h)^{-1} \phi(g) \phi(h) = \phi(h^{-1} g h)$$

      where $h^{-1} g h$ is in $H$, as it is normal. Thus, $\phi(h)^{-1} \phi(g) \phi(h) \in \phi(H)$, so $\phi(H)$ is normal as well. This completes the proof.
    \end{proof}

    \begin{prop}
      Given some cyclic group generated by $x$ of order $n$, and some element $y$ of the group, then $y^n = 1$.
    \end{prop}

    \begin{proof}
      $$y^{n} = (x^{m})^{n} = x^{mn} = (x^{n})^{m} = 1^{m} = 1$$
      \end{proof}

    \begin{problem}{8.3}
      Prove that a finite cyclic group of order $rs$ is isomorphic to the product of cyclic groups of orders $r$ and $s$ if and only if $r$ and $s$ have no
      common factors.
    \end{problem}

    \begin{proof}
      First, assume that $C_{rs}$ is isomorphic to the product of cyclic groups $C_r$ and $C_s$ of orders $r$ and $s$, so there exists a map $\phi : C_{rs} \ \rightarrow \ C_{r} \times C_{s}$.
      Assume that $r$ and $s$ do share a common factor $t$. It follows that $r = at$ and $s = bt$, for some $a$ and $b$.
      \newline

      We then note that $abt = as < rs$, as $a < r$. Since $C_{rs}$ is of order $rs$, then $x^{abt} \neq 1$ for some $x$ in the group.
      \newline

      Clearly, we must have $\phi(x) = (y, \ z)$, for some $y$ and $z$. Thus:

      $$\phi(x^{abt}) = \phi(x)^{abt} = (y, z)^{abt} = (y^{abt}, \ z^{abt}) = ((y^{r})^{b}, \ (z^{s})^{a}) = (1, \ 1)$$

      which contradicts the fact that $\phi$ is injective, as $\phi(1) = (1, \ 1)$ as well.
      \newline

      Now, assume that $r$ and $s$ share no common factors. We define the map that takes $x^{a}$ in $C_{rs}$ to $(y^{a}, \ z^{a})$ in $C_{r} \times C_{s}$.
      \newline

      Clearly, such a map preserves the group operation. In addition, we note that such a rule defines a valid function, as $x^{a + rs} = x^{a}$ will also
      be mapped to $(y^{a}, \ z^{a})$.
      \newline

      Given $(y^{a}, \ z^{b})$ in the codomain, we note that $x^{s} = (y^{s}, \ 1)$, where $y^{s} \neq 1$ and $x^{r} = (1, \ z^{r})$ where $z^{r} \neq 1$. Since the groups
      are cyclic, we then know that there exist powers to which we can raise $y^{s}$ to get $y^{a}$ and $z^{r}$ to get $z^{b}$. Thus, the map is surjective.
      \newline

      In addition, given $x^a$ and $x^b$ such that $\phi(x^a) = \phi(x^b)$. 
      \end{proof}

    \end{document}
