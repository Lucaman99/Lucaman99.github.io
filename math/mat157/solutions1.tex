\documentclass[10pt, oneside]{article} 
\usepackage{amsmath, amsthm, amssymb, calrsfs, wasysym, verbatim, bbm, color, graphics, geometry, hyperref, biblatex, mathtools}
\usepackage[framemethod=TikZ]{mdframed}
\usepackage{tcolorbox}

\hypersetup{
	colorlinks=true,
	linkcolor=blue,
	urlcolor=blue
}

\addbibresource{ref.bib}

\geometry{tmargin=.75in, bmargin=.75in, lmargin=.75in, rmargin = .75in}
\setlength\parindent{0pt}

\tcbuselibrary{theorems}
\newtcbtheorem
  []% init options
  {problem}% name
  {Problem}% title
  {%
    fonttitle=\bfseries,
  }% options
  {prob}% prefix

\newcommand{\R}{\mathbb{R}}
\newcommand{\C}{\mathbb{C}}
\newcommand{\Z}{\mathbb{Z}}
\newcommand{\N}{\mathbb{N}}
\newcommand{\Q}{\mathbb{Q}}
\newcommand{\Cdot}{\boldsymbol{\cdot}}

\newtheorem{thm}{Theorem}
\newtheorem{defn}{Definition}
\newtheorem{conv}{Convention}
\newtheorem{rem}{Remark}
\newtheorem{lem}{Lemma}
\newtheorem{cor}{Corollary}
\newtheorem{prop}{Proposition}

\newcommand{\tr}{\mathrm{Tr}}


\title{Spivak Chapter Problem Set 1 Chapter 28}
\author{Jack Ceroni \thanks{jackceroni@gmail.com}}
\date{September 2020}

\begin{document}

\maketitle
\tableofcontents

\vspace{.25in}

\section{Chapter 28}

\subsection{Problem 5}

\begin{lem}
	For any field, we have:

	$$\underbrace{(e \ + \ \cdots \ + \ e)}_{m \ \text{times}} \ \cdot \ \underbrace{(e \ + \ 
	\cdots \ + \ e)}_{n \ \text{times}} \ = \ \underbrace{(e \ + \ \cdots \ + \ e)}_{mn \ \text{times}}$$

	for all natural numbers $n$ and $m$.
\end{lem}

\begin{proof}
	Pick ome arbitrary natural number $m$. We proceed by induction. Clearly, the lemma is true in 
	the case of $n \ = \ 1$. Let us assume the case of $n$. Consider the case of $n \ + \ 1$. We have:
	
	$$\underbrace{(e \ + \ \cdots \ + \ e)}_{m \ \text{times}} \ \cdot \ \underbrace{(e \ + \ 
	\cdots \ + \ e)}_{n \ + \ 1 \ \text{times}} \ = \ \underbrace{(e \ + \ \cdots \ + \ e)}_{m \ \text{times}} \ \cdot \ \big[ \underbrace{(e \ + \ 
\cdots \ + \ e)}_{n \ \text{times}} \ + \ e \big]$$

  Now, we use the distributive property of fields and the definition of the identity, along with the assumtpion that
  the lemma holds true in the case of $n$ to get:

$$\Rightarrow \ \big[ \underbrace{(e \ + \ 
\cdots \ + \ e)}_{mn \ \text{times}} \ + \ e \ \cdot \  \underbrace{(e \ + \ \cdots \ + \ e)}_{m \ \text{times}} \big] \ = \ 
\big[ \underbrace{(e \ + \ 
\cdots \ + \ e)}_{mn \ \text{times}} \ + \ \underbrace{(e \ + \ \cdots \ + \ e)}_{m \ \text{times}} \big] \ = \ 
\underbrace{(e \ + \ 
\cdots \ + \ e)}_{m(n \ + \ 1) \ \text{times}}$$ 

So the lemma is proved.

\end{proof}

\begin{thm}
	If in some field $F$ we have:

	$$\underbrace{e \ + \ \cdots \ + \ e}_{n \ \text{times}} \ = \ 0$$

	then the smallest $n$ for which this is true is prime.
\end{thm}

\begin{proof}
	Assume that $n$ isn't prime. It follows that we can write $n$ as a product of at least two whole numbers 
	less than $n$ and greater than $1$. Thus, $n \ = \ ab$. By the previous lemma, we have:
\newpage	
	$$\underbrace{e \ + \ \cdots \ + \ e}_{n \ \text{times}} \ = \ \underbrace{(e \ + \ \cdots \ + \ e)}_{a \ \text{times}} \ \cdot \ \underbrace{(e \ + \ 
	  \cdots \ + \ e)}_{b \ \text{times}} \ = \ 0$$

  In a field, we know that $a \ \cdot \ 0 \ = \ a$, as $0$ is the element of the field such that $a \ + \ 0 \ = \ a$. We then have (by distribution) that
  $(a \cdot a) \ + \ (a \cdot 0) \ = \ (a \cdot a) \ = \ (a \cdot a) \ + \ 0$. By left cancellation, we have $a \cdot 0 \ = \ 0$. Assume that both the right-hand sums
  of $e$ (for $a$ and $b$) are non-zero. It follows that they have inverses. Let us denote the two sums by $A$ and $B$. It follows that:

  $$e \ = \ A^{-1} A B^{-1} B \ = \ (A^{-1} B^{-1}) \cdot (A B) \ = \ (A^{-1} B^{-1}) \cdot 0 \ = \ 0$$

  which is a contradiction to the definition of a field, as the additive and multiplicative identities must be different. Thus, at least one of these sums is equal to $0$ it follows that either $a$ or $b$ is a whole number
  less than $n$ such that:

  $$\underbrace{e \ + \ \cdots \ + \ e}_{a \ \text{or} \ b \ \text{times}} \ = \ 0$$

  which is a contradiction. Thus, $n$ must be prime.

\end{proof}

\subsection{Problem 6}

\begin{lem}
  For some field $F$ with a finite number of elements, there exist distinct natural numbers $m$ and $n$ such that:

  $$\underbrace{e \ + \ \cdots \ + \ e}_{m \ \text{times}} \ = \ \underbrace{e \ + \ \cdots \ + \ e}_{n \ \text{times}}$$
\end{lem}

\begin{proof}
  Let $|F| \ = \ k$ be the cardinality of the set defining the field (which we know is some finite natural number, $k$). Let:

  $$E(n) \ = \ \underbrace{e \ + \ \cdots \ + \ e}_{n \ \text{times}}$$

  Now, consider the set $\{E(1), \ E(2), \ ..., \ E(k), \ E(k \ + \ 1)\}$. It follows that there must exist two elements of
  this set that are equal, or else we would have a subset of $F$ that contains $k \ + \ 1$ \textbf{distinct} elements, a clear contradiction.
  Hence, there exist $m$ and $n$ such that:

  $$\underbrace{e \ + \ \cdots \ + \ e}_{m \ \text{times}} \ = \ \underbrace{e \ + \ \cdots \ + \ e}_{n \ \text{times}}$$

\end{proof}

\begin{thm}
  In a field $F$ with a finite number of elements, there exists some natural number $r$ such that:

  $$\underbrace{e \ + \ \cdots \ + \ e}_{r \ \text{times}} \ = \ 0$$
\end{thm}

\begin{proof}
  By the previous lemma, we know there exist $m$ and $n$ such that $E(m) \ = \ E(n)$. Without loss of generality, let $n \ < \ m$ (the two numbers
  are distinct, so one is larger than the other). We have:

  $$0 \ + \ \underbrace{e \ + \ \cdots \ + \ e}_{n \ \text{times}} \ = \ \underbrace{e \ + \ \cdots \ + \ e}_{m \ \text{times}} \ = \ \underbrace{e \ + \ \cdots \ + \ e}_{m \ - \ n \ \text{times}} \ + \
  \underbrace{e \ + \ \cdots \ + \ e}_{n \ \text{times}}$$

  So by right cancellation, we have:

  $$\underbrace{e \ + \ \cdots \ + \ e}_{m \ - \ n \ \text{times}} \ = \ 0$$

  It follows that $r \ = \ m \ - \ n$ and the theorem is proved.
  \end{proof}

\end{document}
