\documentclass[10pt, oneside]{article} 
\usepackage{amsmath, amsthm, amssymb, calrsfs, wasysym, verbatim, bbm, color, graphics, geometry, hyperref, biblatex, mathtools}
\usepackage[framemethod=TikZ]{mdframed}
\usepackage{tcolorbox}

\hypersetup{
	colorlinks=true,
	linkcolor=blue,
	urlcolor=blue
}

\addbibresource{ref.bib}

\geometry{tmargin=.75in, bmargin=.75in, lmargin=.75in, rmargin = .75in}
\setlength\parindent{0pt}

\tcbuselibrary{theorems}
\newtcbtheorem
    []% init options
    {problem}% name
    {Problem}% title
    {%
      fonttitle=\bfseries,
    }% options
    {prob}% prefix

    \newcommand{\R}{\mathbb{R}}
    \newcommand{\C}{\mathbb{C}}
    \newcommand{\Z}{\mathbb{Z}}
    \newcommand{\N}{\mathbb{N}}
    \newcommand{\Q}{\mathbb{Q}}
    \newcommand{\Cdot}{\boldsymbol{\cdot}}

    \newtheorem{thm}{Theorem}
    \newtheorem{defn}{Definition}
    \newtheorem{conv}{Convention}
    \newtheorem{rem}{Remark}
    \newtheorem{lem}{Lemma}
    \newtheorem{cor}{Corollary}
    \newtheorem{prop}{Proposition}

    \newcommand{\tr}{\mathrm{Tr}}


    \title{Running Putnam Notes}
    \author{Jack Ceroni}
    \date{October 2020}

    \begin{document}

    \maketitle
    \tableofcontents

    \vspace{.25in}

    
    \section{Introduction}

    \textit{These notes are my attempt at keeping a log of any useful Putnam-related notes and solutions during undergrad.}

    \section{Solutions}

    \begin{prop}
      Every non-zero coefficient of the Maclauren series of:

      $$f(x) = (1 - x + x^2) e^x$$

      is rational, with the numerator being either $1$ or a prime number (when fully reduced).
    \end{prop}

    \begin{proof}
      Clearly:

      \begin{equation}
        \begin{split}
        f(x) = (1 - x + x^2) e^x = (1 - x + x^2) \displaystyle\sum_{n = 0}^{\infty} \frac{x^n}{n!} = \displaystyle\sum_{n = 0}^{\infty} \frac{x^n - x^{n + 1} + x^{n + 2}}{n!} \\
         = \displaystyle\sum_{n = 0}^{\infty} \frac{x^n}{n!} - \displaystyle\sum_{n = 1}^{\infty} \frac{x^n}{(n - 1)!} + \displaystyle\sum_{n = 2}^{\infty} \frac{x^n}{(n - 2)!}
        \end{split}
      \end{equation}

      In the case that $n \geq 2$, the $n$-th coefficient of the series(which we denote by $c_n$) is thus given by:

      \begin{equation}
        c_n = \frac{1}{n!} - \frac{1}{(n + 1)!} + \frac{1}{(n + 2)!} = \frac{1 - n + n(n + 1)}{n!} = \frac{n^2 + 1}{n!}
        \end{equation}
    \end{proof}
    
    \end{document}
