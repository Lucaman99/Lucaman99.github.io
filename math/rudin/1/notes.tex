\documentclass[10pt, oneside]{article} 
\usepackage{amsmath, amsthm, amssymb, calrsfs, wasysym, verbatim, bbm, color, graphics, geometry, hyperref, biblatex, mathtools}

\hypersetup{
	colorlinks=true,
	linkcolor=blue,
	urlcolor=blue
}

\addbibresource{ref.bib}

\geometry{tmargin=.75in, bmargin=.75in, lmargin=.75in, rmargin = .75in}  

\newcommand{\R}{\mathbb{R}}
\newcommand{\C}{\mathbb{C}}
\newcommand{\Z}{\mathbb{Z}}
\newcommand{\N}{\mathbb{N}}
\newcommand{\Q}{\mathbb{Q}}
\newcommand{\Cdot}{\boldsymbol{\cdot}}

\newtheorem{thm}{Theorem}
\newtheorem{defn}{Definition}
\newtheorem{conv}{Convention}
\newtheorem{rem}{Remark}
\newtheorem{lem}{Lemma}
\newtheorem{cor}{Corollary}
\newtheorem{prop}{Proposition}

\newcommand{\tr}{\mathrm{Tr}}
\setlength\parindent{0pt}
\renewcommand{\qedsymbol}{$\blacksquare$}


\title{Baby Rudin Chapter 1 Notes}
\author{Jack Ceroni}
\date{September 2020}

\begin{document}

\maketitle
\tableofcontents

\vspace{.25in}

\section{Motivation for Writing Notes}

\textit{The reason I'm writing these notes is to better understand certain aspects of Rudin's
	book on real analysis. By writing things in my own words and going through 
	concepts that I don't fully understand in a systemic fashion, I'm confident my 
	understanding of the material will be much deeper by the time I'm done reading this book.}

\section{Introduction}

\begin{thm}
The number $\sqrt{2}$ is irational.
\end{thm}

\begin{proof}
	By definition, $\sqrt{2}$ is the number $p$ that satisfies the 
	equation:

	$$p^2 \ = \ 2$$

	Assume that $p$ is rational, so it can be written as an irreducable 
	quotient of integers $p \ = \ m/n$. It follows that:

	$$m^2 \ = \ 2n^2$$

	This implies that $m^2$ is even, so $m$ must be even (an odd times and odd is equal
	to an odd). Let us write $m$ as $2a$. We thus have:

	$$4a^2 \ = \ 2n^2 \ \Rightarrow \ n^2 \ = \ 2a^2$$

	it follows that $n$ is also even. Let us write $n$ as $2b$. Thus, we have:

	$$p \ = \ \frac{m}{n} \ = \ \frac{2a}{2b} \ = \ \frac{a}{b}$$

	Thus, $m$ and $n$ do not form a irreducable fraction, contrary to our initial 
	assumption. This implies that $p$ cannot be written as such a fraction and is thus 
	irrational.
\end{proof}

\begin{rem}
  There is a decent bit to be said about how Rudin went about choosing the formula used to demonstrate that the sets of rationals with
  $p^2 < 2$ and $p^2 > 2$ have no greatest and smallest element. However, I'll probably circle back and fill this section in when
  I have two to draw some graphs to really build visual intuition.
\end{rem}

\subsection{The Real Field}

\begin{thm}
	If $x, \ y \in \mathbb{R}$, and $x > 0$, then there is a positive integer 
	$n$ such that:

	$$nx > y$$
\end{thm}

\begin{rem}
	The idea behind this proof is to derive a contradiction by showing that some subset 
	of the reals with an upper bound has no least upper bound, as we are only working with 
	that assumption.
\end{rem}

\begin{proof}
	Assume that there existed some pair $(x, y)$ such that there existed no positive 
	integer $n$ such that $nx > y$. It follows that for any integer $n$:

	$$nx \ \leq \ y$$

	It follows that $y$ is an upper bound of the set of all $nx$, for integer $n$. By definition of $\mathbb{R}$,
  this set has a least upper bound. Let $m$ be this least-upper bound:

  $$nx \ \leq \ m$$

  Consider the number $m - x$. There must be some $nx > m - x$, or else $m - x$ would be a lower upper
  bound than $m$ oon the set of all $nx$. It follows that $(n + 1)x > m$. Since $n + 1$ is an integer, this
  is a contradiction. Thus, no such $m$ exists. It follows that there must exist some $n$ such that $nx > y$, for the
  pair $x$ and $y$.
\end{proof}

\begin{thm}
  If $x \in \mathbb{R}$ and $y \in \mathbb{R}$, and $x < y$, then there exists a $p \in \mathbb{Q}$ such that $x < p < y$.
\end{thm}

\begin{rem}
  This proof seemed significantly more difficult than the previous one, so I will simply be making remarks on Rudin's proof rather than attempt it myself.
  The general idea behind this proof is that we wish to scale both $x$ and $y$ by some integer $n$, until they are far enough apart such that there is an integer $m$ between
  them. Dividing by $n$ then gives $m/n$ between $x$ and $y$.
  \newline\newline
  The real question is how we go about finding an $n$ that allows for this. Well, since $x  - y > 0$, we can choose some $n$ such that
  $n(x - y) > 1$. If $x - y > 1$, then $n$ can just be one and if $1 > x - y$, then we apply the previous theorem. Conceptually, it makes
  sense that we should be able to find an integer $m$ in this range. If we're given a number line and mark two points that are separated by a distance greater than $1$,
  that interval should overlap with an integer.
  \newline\newline
  To find this integer, We make use of the fact that there is some $m_1 > nx$ (by the previous theorem). Similarly, we can prove there is an integer below $nx$ by noting that
  there is some $m_2 > -nx \ \Rightarrow \ -m_2 < nx$. Since $nx$ is located between two integers, it follows that it must be between two succesive integers, $m$ and $m - 1$.
  \newline\newline
  Now, we just need to show that $m < ny$. This follows from the fact that the distance between $nx$ and $ny$ is greater than $1$:

  $$nx < m < nx + 1 < ny$$

  So there is an $m$ between $nx$ and $ny$. It follows that:

  $$x < \frac{m}{n} < y$$
\end{rem}

\begin{rem}
	For the moment, we will not discuss the "power-uniqueness" proof.
\end{rem}

\section{Basic "Topology"}

\textit{The reason I put topology in quotes is because we are technically only studying topology on the real line and 
products of the real line.}

\subsection{Finite and Infinite Sets}

\begin{thm}
	Every infinite subset of a countable set is countable.
\end{thm}

\begin{rem}
	Rudin presents a non-rigorous version of the proof of this fact. The "adult" version requires the well-ordering principle
	of the integers and the principle of recursive definition.
\end{rem}

\begin{thm}
	A set is open if and only if its complement is closed.
\end{thm}

\begin{proof}
	Consider the set $U$ in $X$. Let $U$ be open. It follows that for each $x \ \in \ U$, there exists some $N_r(x)$ such that 
	$x \ \in \ N_r(x) \ \subset \ U$. Consider the complement of $U$, which we denote by $X \ - \ U$. Assume that there is a 
	limit point $y$ of $X \ - \ U$ that is in $U$. Thus, there must exist some neighbourhood around $x$ that is contained in 
	$U$, which contradicts the fact that $y$ is a limit point. Thus, $X \ - \ U$ must contain all of its limit points and 
	is closed.
	\newline\newline
	Conversely, assume $X \ - \ U$ is closed, and thus contains all of its limit points. Assume there is some $x \ \in \ U$ 
	that is contained in no neighbourhood that is a subset of $U$. It follows that each neighbourhood of $x$ interssects 
	$X \ - \ U$, making it a limit point. This is a contradiction, so each point of $U$ must be an interior point, making 
	$U$ open.
\end{proof}

\end{document}
