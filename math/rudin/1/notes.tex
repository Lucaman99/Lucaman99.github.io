\documentclass[10pt, oneside]{article} 
\usepackage{amsmath, amsthm, amssymb, calrsfs, wasysym, verbatim, bbm, color, graphics, geometry, hyperref, biblatex, mathtools}

\hypersetup{
	colorlinks=true,
	linkcolor=blue,
	urlcolor=blue
}

\addbibresource{ref.bib}

\geometry{tmargin=.75in, bmargin=.75in, lmargin=.75in, rmargin = .75in}  

\newcommand{\R}{\mathbb{R}}
\newcommand{\C}{\mathbb{C}}
\newcommand{\Z}{\mathbb{Z}}
\newcommand{\N}{\mathbb{N}}
\newcommand{\Q}{\mathbb{Q}}
\newcommand{\Cdot}{\boldsymbol{\cdot}}

\newtheorem{thm}{Theorem}
\newtheorem{defn}{Definition}
\newtheorem{conv}{Convention}
\newtheorem{rem}{Remark}
\newtheorem{lem}{Lemma}
\newtheorem{cor}{Corollary}
\newtheorem{prop}{Proposition}

\newcommand{\tr}{\mathrm{Tr}}
\setlength\parindent{0pt}
\renewcommand{\qedsymbol}{$\blacksquare$}


\title{Baby Rudin Chapter 1 Notes}
\author{Jack Ceroni}
\date{September 2020}

\begin{document}

\maketitle
\tableofcontents

\vspace{.25in}

\section{Motivation for Writing Notes}

\textit{The reason I'm writing these notes is to better understand certain aspects of Rudin's
	book on real analysis. By writing things in my own words and going through 
	concepts that I don't fully understand in a systemic fashion, I'm confident my 
	understanding of the material will be much deeper by the time I'm done reading this book.}

\section{Introduction}

\begin{thm}
The number $\sqrt{2}$ is irational.
\end{thm}

\begin{proof}
	By definition, $\sqrt{2}$ is the number $p$ that satisfies the 
	equation:

	$$p^2 \ = \ 2$$

	Assume that $p$ is rational, so it can be written as an irreducable 
	quotient of integers $p \ = \ m/n$. It follows that:

	$$m^2 \ = \ 2n^2$$

	This implies that $m^2$ is even, so $m$ must be even (an odd times and odd is equal
	to an odd). Let us write $m$ as $2a$. We thus have:

	$$4a^2 \ = \ 2n^2 \ \Rightarrow \ n^2 \ = \ 2a^2$$

	it follows that $n$ is also even. Let us write $n$ as $2b$. Thus, we have:

	$$p \ = \ \frac{m}{n} \ = \ \frac{2a}{2b} \ = \ \frac{a}{b}$$

	Thus, $m$ and $n$ do not form a irreducable fraction, contrary to our initial 
	assumption. This implies that $p$ cannot be written as such a fraction and is thus 
	irrational.
\end{proof}

\begin{rem}
There is a decent bit to be said about how Rudin went about choosing the formula used to demonstrate that the sets of rationals with $p^2 \ < \ 2$ and $p^2 \ > \ 2$ have no greatest and smallest element. However, I'll probably circle back and fill this section in when I have two to draw some graphs to really build visual intuition.
\end{rem}

\section{The Real Field}

\begin{thm}
	If $x, \ y \ \in \ \mathbb{R}$, and $x \ > \ 0$, then there is a positive integer 
	$n$ such that:

	$$nx \ > \ y$$
\end{thm}

\begin{rem}
	The idea behind this proof is to derive a contradiction by showing that some subset 
	of the reals with an upper bound has no least upper bound, as we are only working with 
	that assumption.
\end{rem}

\begin{proof}
	Assume that there existed some pair $(x, \ y)$ such that there existed no positive 
	integer $n$ such that $nx \ > \ y$. It follows that for any integer $n$:

	$$nx \ < \ y$$

	It follows that $y$ is an upper bound of the set of all $nx$, for integer $n$.  
\end{proof}


\end{document}
