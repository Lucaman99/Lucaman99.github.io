\documentclass[aps,pra,showpacs,notitlepage,onecolumn,superscriptaddress,nofootinbib]{revtex4-1}
\usepackage[utf8]{inputenc}
\usepackage[tmargin=1in, bmargin=1.25in, lmargin=1.5in, rmargin=1.5in]{geometry}
\usepackage{amsmath, amssymb, amsthm}
\usepackage{graphicx}
\usepackage{xcolor}
\usepackage{enumitem}
\usepackage{datetime}
\usepackage{hyperref}
\usepackage{titlesec}
\usepackage{import}
\usepackage{mathtools}
\usepackage{thmtools,thm-restate}
\usepackage{tikz-cd}
\usepackage[many]{tcolorbox}

% package for commutative diagrams
% \usepackage{tikz-cd}

%%%%%%%%%%%%%%%%%%%%%%%%%%%%%%%%%%%%%%%%%%%%%
\definecolor{crimson}{RGB}{186,0,44}
\definecolor{moss}{RGB}{0, 186, 111}
\newcommand{\pop}[1]{\textcolor{crimson}{#1}}
\newcommand{\zcom}[1]{\noindent\textcolor{crimson}{(Z): #1}}
\newcommand{\jcom}[1]{\noindent\textcolor{moss}{(J): #1}}
\newcommand{\wt}[1]{\widetilde{#1}}
\newcommand{\pqeq}{\succcurlyeq}
\newcommand{\pleq}{\preccurlyeq}

%%%%%%%%%%%%%%%%%%%%%%%%%%%%%%%%%%%%%%%%%%%%%
\hypersetup{
    colorlinks,
    linkcolor={crimson},
    citecolor={crimson},
    urlcolor={crimson}
}

\usepackage{qcircuit}
\usepackage{comment}

%%%%%%%%%%%%%%%%%%%%%%%%%%%%%%%%%%%%%%%%%%%%%
\theoremstyle{definition}
\newtheorem{definition}{Definition}[section]

\newtheorem{lemma}{Lemma}[section]

\newtheorem{theorem}{Theorem}[section]

\newtheorem{corollary}{Corollary}[theorem]
\newtheorem*{theorem*}{Theorem}
\newtheorem*{corollary*}{Corollary}

\newtheorem{remark}{Remark}[section]

\newtheorem{conjecture}{Conjecture}[section]
\newtheorem{example}{Example}[section]
\newtheorem{reminder}{Reminder}[section]
\newtheorem{problem}{Problem}[section]
\newtheorem{question}{Question}[section]
\newtheorem{answer}{Answer}[section]
\newtheorem{fact}{Fact}[section]
\newtheorem{claim}{Claim}[section]
\newtheorem{prop}{Proposition}[section]

\newtheorem{solution}{Solution}[section]

\usepackage{geometry}
\geometry{
  left=25mm,
  right=25mm,
  top=20mm,
}

\newcommand{\hhrulefill}{\hspace{-1.5em} \hrulefill}
\renewcommand{\baselinestretch}{1.1} 

%%%%%%%%%%%%%%%%%%%%%%%%%%%%%%%%%%%%%%%%%%%%%
\bibliographystyle{unsrt}

%%%%%%%%%%%%%%%%%%%%%%%%%%%%%%%%%%%%%%%%%%%%%
%%%%%%%%%%%%%%%%%%%%%%%%%%%%%%%%%%%%%%%%%%%%%
%%%%%%%%%%%%%%%%%%%%%%%%%%%%%%%%%%%%%%%%%%%%%
\begin{document}

\title{Every exercise in the first three chapters of Hartshorne}
\author{Jack Ceroni}
\email{jceroni@uchicago.edu}
\date{\today}
\maketitle

\tableofcontents

\section{Introduction}

\noindent Every exercise in the first three chapters of Hartshorne, hopefully, eventually (``every'' means \textbf{every single one}).
\newline

\noindent \textbf{Current tally}

\noindent There are some partially-completed problems in the document as well, but these are not included in the tally.
\newline

\begin{itemize}
  \item Chapter 1: 16/90
    \item Chapter 2: 9/134
      \item Chapter 3: 0/88
  \end{itemize}

\section{Chapter 1}

\subsection{Section 1.1}

\begin{solution}[Problem 1.1.1]
There are a few parts:
\begin{enumerate}
  \item Of course, $A(Y) = k[x, y]/(y - x^2)$. We can define $\varphi : k[x, y] \rightarrow k[x]$ as $\varphi(p)(x) = p(x, x^2)$. Verification that this is a ring homomorphism
    is trivial. It is obviously surjective, as $k[x] \subset k[x, y]$, and $\varphi|_{k[x]} = \text{id}$. In addition, $\varphi(y - x^2) = 0$. Moreover, if $\varphi(p) = 0$, then
    $p(x, x^2) = 0$. Define $h(x, y) = p(x, y + x^2) \in k[x, y] = k[x][y]$. Of course, we may write $h(x, y) = h_0(x) + h_1(x) y + h_2(x) y^2 + \cdots$, by definition, and $p(x, x^2) = h(x, 0) = h_0(x)$,
    so $h(x, y) = y g(x, y)$ for some $g$. Thus, $p(x, y) = (y - x^2) g(x, y - x^2)$, which means $p \in (y - x^2)$. It follows that $\text{Ker}(\varphi) = (y - x^2)$, so by the first isomorphism
    theorem, $k[x, y]/(y - x^2) \simeq \text{Im}(\varphi) = k[x]$, so $A(Y) \simeq k[x]$, as desired.
    \item Suppose $\varphi : k[x, y]/(xy - 1) \rightarrow k[x]$ is a ring homomorphism. Since $[x][y] = [xy] = [1]$ in the domain, $\varphi([x]) \varphi([y]) = \varphi([x][y]) = \varphi([1]) = 1$. Thus,
      $\varphi([x]) = a \in k$ and $\varphi([y]) = a^{-1} \in k$. Given some $b \in k$, with $b \neq 0$, note that we must also have $\varphi([b]) \varphi([b^{-1}]) = 1$, so $\varphi([b])$ is a unit in $k[x]$,
      thus in $k$. It follows that $\text{Im}(\varphi) = k$, which means that $\varphi$ cannot be an isomorphism.
     \item \textbf{(Starred)} The idea is to make use of a sequence of automorphisms of $k[x, y]$, which descend to automorphisms of the quotient. In particular, a general quadratic polynomial is of the form
       \begin{equation}
         p(x, y) = a x^2 + b y^2 + cxy + dx + ey + f
         \end{equation}
       Since we are working over an algebraically closed field, we are allowed to take square roots. For this to be a quadratic, at least one of $a, b$ or $c$ must be non-zero. Assume $c$ is non-zero, and
       assume that $4ab = c^2$. It then follows that either $a$ or $b$ is non-zero (we can assume WLOG it is $a$), so the map $(x, y) \mapsto (\sqrt{a} x + \sqrt{b} y, y)$ is an isomorphism. Note that
       we can write
       \begin{equation}
         ax^2 + by^2 + cxy + dx + ey + f = (\sqrt{a} x + \sqrt{b} y)^2 + dx + ey + f
         \end{equation}
       so it follows that we may reduce to the case that our quadratic is of the form $x^2 + dx + ey + f$. From here, note that $e \neq 0$, or else our polynomial in $x$ would be reducible. In addition,
       we can re-write $x^2 + dx + f$ in the form $-(i x - d')^2 + f'$, so that
       \begin{equation}
         x^2 + dx + ey + f = (ey + f') - (i x - d')^2
         \end{equation}
       and note that $(x, y) \mapsto (ix - d', ey + f')$ is an isomorphism, so we have reduced to the case where our quadratic is $y - x^2$.

       Next, consider the case where $4ab \neq c^2$, still with $c$ non-zero and also with $a \neq 0$ or $b \neq 0$.
       The idea is that we can transform $ax^2 + by^2 + cxy$ to $xy$. In particular, assume WLOG that $a \neq 0$, so we can factor out a unit
       and assume the degree-$2$ part of our quadratic is of the form
       \begin{equation}
         x^2 + by^2 + cxy = (x - r_{+} y)(x - r_{-} y)
         \end{equation}
       with $4b \neq c^2$, where
       \begin{equation}
         r_{\pm} = \frac{c \pm \sqrt{c^2 - 4b}}{2}
         \end{equation}
       Note that $r_{+} \neq r_{-}$, as their difference is $\sqrt{c^2 - 4b}$, which is non-zero. Thus, at least one of them is non-zero, so that $(x, y) \mapsto (x - r_{+} y, x - r_{-} y)$
       is an isomorphism. It follows that we can reduce our quadratic to the form $xy + dx + ey + f$. This is also the form of the quadratic, after factoring out $c$, if we didn't first assume
       that either $a$ or $b$ is non-zero. Of course, $xy + dx + ey + f = (x + e)(y + d) + f$, so we can reduce to the case $xy + f$. If $f = 0$, our quadratic is reducible, which is a contradiction,
       so $f \neq 0$. Factoring out a unit reduces to the case $xy - 1$.

       The final case that we must check is when $c = 0$. Again, either $a$ or $b$ must be non-zero, or we don't have a quadratic. If one of these \emph{is} zero, then our quadratic can be assumed
       of the form $x^2 + dx + ey + f$: the same as in the first case worked out, so this will reduce to the case of $y - x^2$. If both are non-zero, then we have
       \begin{equation}
         ax^2 + by^2 = (\sqrt{a} x - i \sqrt{b} y)(\sqrt{a} x + i \sqrt{b} y)
         \end{equation}
       with $(x, y) \mapsto (\sqrt{a} x - i \sqrt{b} y, \sqrt{a} x + i \sqrt{b} y)$ being an isomorphism, reducing us to the case of $xy + dx + ey + f$: another case which we already worked out.

       Thus, to summarize, given some irreducible quadratic, we can always find a sequence of ring automorphisms of $k[x, y]$ and multiplication by units which takes the quadratic to
       either $y - x^2$ or $xy - 1$. Given a ring automorphism $\varphi : k[x, y] \rightarrow k[x, y]$, this will descend to a ring automorphism $\widetilde{\varphi} : k[x, y]/(f) \rightarrow k[x, y]/(\varphi(f))$
       where $\widetilde{\varphi}(g + (f)) = \varphi(g) + (\varphi(f))$. In addition, if $c$ is a unit, $(f)$ and $(cf)$ are the same ideal. It follows that via our sequence of transformations, we can exhibit
       an isomorphism which takes $k[x, y]/(f)$ to $k[x, y]/(y - x^2)$ or $k[x, y]/(xy - 1)$, with the individual criteria for each case explained in the course of the solution above.
  \end{enumerate}
  \end{solution}

\begin{solution}[Problem 1.1.2]
  Clearly, $Y = V(y - x^2, z - x^3)$. We want to show that $(y - x^2, z - x^3)$ is prime. To do so, define the map $\varphi : k[x, y, z] \rightarrow k[x]$ which takes $p(x, y, z)$ to $p(x, x^2, x^3)$. It is
  clear that this is a surjective ring homomorphism and $(y - x^2, z - x^3) \subset \text{Ker}(\varphi)$. Moreover, suppose $\varphi(p) = 0$. Define $h(x, y, z) = p(x, y + x^2, z + x^3)$. Of course,
  $h(x, 0, 0) = 0$, so we can write
  \begin{equation}
    h(x, y, z) = y g_1(x, y, z) + z g_2(x, y, z)
    \end{equation}
  as each term is divisible by $z$ or $y$. We then have
  \begin{equation}
    p(x, y, z) = h(x, y - x^2, z - x^3) = (y - x^2) g_1(x, y - x^2, z - x^3) + (z - y^3) g_2(x, y - x^2, z - x^3)
    \end{equation}
  which means that $p \in (y - x^2, z - x^3)$. Thus, $k[x] \simeq k[x, y, z]/(y - x^2, z - x^3)$ so $(y - x^2, z - x^3)$ is prime, so $Y$ is an affine variety and is $1$-dimensional as $\dim(k[x]) = 1$.
  We have $I(Y) = (y - x^2, z - x^3)$, so we automatically have a pair of generators (this ideal clearly cannot be generated by one of these generators).
  \end{solution}

\begin{solution}[Problem 1.1.3]
  We have
  \begin{align}
    V(x^2 - yz, xz - x) &= V(x^2 - yz) \cap [V(x) \cup V(z - 1)]
    \\ &= V(x^2 - yz, x) \cup V(x^2 - yz, z - 1)
    \\ &= V(yz, x) \cup V(x^2 - y, z - 1)
    \\ &= ([V(y) \cup V(z)] \cap V(x)) \cup V(x^2 - y, z - 1)
    \\ &= V(y, x) \cup V(z, x) \cup V(y - x^2, z - 1)
    \end{align}
  Showing that each of the ideals $(y, x)$, $(z, x)$ and $(y - x^2, z - 1)$ can be done easily via the same method as Problem 1.1.2. Thus, $Y$ is the union of three irreducible components.
  It is easy to see that they each are $1$-dimensional (as their coordinate rings are isomorphic to polynomials in a single variable).
  \end{solution}

\begin{solution}[Problem 1.1.4]
  Every open set in the product topology for $\mathbb{A}^1 \times \mathbb{A}^1$ can be written as the union of a collection of $U \times V$, with $U, V \subset \mathbb{A}^1$ open. Of course, $U$ and $V$
  will be $\mathbb{A}^1$ with some finite collection of points removed (when they are non-empty). Thus, $U \times V$ is $\mathbb{A}^2$ with some finite collection of vertical/horizontal lines removed.
  All of these sets are open in the Zariski topology for $\mathbb{A}^2$. However, the Zariski topology is strictly finer than the product topology.

  In particular, consider the open set $Y = V(x - y)^{C}$ in the Zariski topology. Given some point $p$ in $Y$, if $Y$ is open in the product topology we must be able to choose some $U \times V$
  which is a neighbourhood of $p$ and contained in $Y$. However, any complement of a finite collection of horizontal/vertical lines will clearly contain some point on the diagonal $V(x - y)$.
  \end{solution}

\begin{solution}[Problem 1.1.5]
  First, note that the coordinate ring of some algebraic set $V$ is of the form $A(V) = k[x_1, \dots, x_n]/I(V)$, where $I(V)$ is a radical ideal. It follows immediately that if $[p]^n = [p^n] = 0$ in $A(V)$,
  then $p^n \in I(V)$, so $p \in I(V)$ and $[p] = 0$. This means $A(V)$ has no nilpotent elements. Moreover, it is immediately clear that $A(V)$ is a finitely-generated $k$-algebra, generated by $[x_1], \dots, [x_n]$.

  Conversely, suppose $B$ is a finitely-generated $k$-algebra with no nilpotent elements. Let $b_1, \dots, b_n$ be a generating set for $B$, then the evaluation map $\varphi : k[x_1, \dots, x_n] \rightarrow B$
  sending $p$ to $p(b_1, \dots, b_n)$ is surjective. It follows that $B \simeq k[x_1, \dots, x_n]/\text{Ker}(\varphi)$. Denote the ideal $\text{Ker}(\varphi)$ by $I$. Since $B$ has no nilpotent elements, it follows
  that if $p^n \in I$ so that $p(b_1, \dots, b_n)^n = 0$, then $p(b_1, \dots, b_n) = 0$, so $p \in I$. Thus, $I$ is a radical ideal and $I(V(I)) = I$. It follows that $B \simeq A(V(I))$.
  \end{solution}

\begin{solution}[Problem 1.1.6]
If $X$ is a topological space with non-empty open set $U$ and $\overline{U}$ is proper, then $X = \overline{U} \cup (X - U)$, so $X$ is reducible. Thus, if $X$ is irreducible, every non-empty open
set is dense. For the second part, suppose $C_1$ and $C_2$ are closed in $\overline{Y}$ with $C_1 \cup C_2 = \overline{Y}$. Then since $\overline{Y}$ is closed in $X$, $C_1$ and $C_2$ are as well. Thus,
$C_1 \cap Y$ and $C_2 \cap Y$ is closed in $Y$. Since $Y$ is irreducible and the union of these closed sets is $Y$, either $Y \subset C_1$ or $Y \subset C_2$. Thus, $\overline{Y} \subset C_1$ or $\overline{Y} \subset C_2$,
so $\overline{Y}$ is irreducible.
  \end{solution}

\begin{solution}[Problem 1.1.7]
  This is a long problem:
  \begin{enumerate}
    \item Given some non-empty family $\mathcal{F}$ of closed sets, suppose there is no minimal element. Let $C_1 \in \mathcal{F}$, we can choose
      $C_2 \in \mathcal{F}$ which is a proper subset of $C_1$. We continue on inductively, choosing $C_{j + 1} \in \mathcal{F}$ a proper subset of $C_j$. At follows that
      we have a descending chain in $X$ which does not terminate, so $X$ is not Noetherian. On the other hand, given a descending chain $C_1 \supset C_2 \supset \cdots$, there must
      be a minimal element (as such a chain is a family), hence some $C_n$ such that if $C_j \subset C_n$, then $C_n = C_j$, so the chain terminates after a finite number of steps. The open set conditions can easily
      be seen to be equivalent to the conditions above by taking complements.
      \item Let $\{U_{\alpha}\}_{\alpha}$ be an open cover for $X$. Suppose there is no finite subcover. Pick $U_1$ in the cover arbitrarily. Then pick some $x_2 \in U_1^{C}$ and pick a neighbourhood
        of $x_2$ in the cover, label it $U_2$. We proceed inductively, choosing $x_{n + 1} \in (U_1 \cup \cdots \cup U_n)^{C}$, letting $U_{n + 1} \in \{U_{\alpha}\}$ being a neighbourhood of $x_{n + 1}$. Then the ascending
        chain of open set $V_n = U_1 \cup \cdots \cup U_n$ never terminates, as $V_{n + 1}$ contains $x_{n + 1}$ while $V_n$ does not. Thus, $X$ is not Noetherian. It follows that if $X$ is Noetherian, then it must be
        quasi-compact.
        \item Any descending chain $C_1 \supset C_2 \supset \cdots$ of closed sets of $Y \subset X$ implies that $C_j = K_j \cap Y$ with $K_j$ a closed set. We then define
          $K'_n = K_1 \cap \cdots \cap K_n$, which gives a descending chain $K_1' \supset K_2' \supset \cdots$ of $X$. This chain eventually terminates, so $K_n' = K_{n + 1}' = \cdots$.
          Note that
          \begin{equation}
          K_j' \cap Y = (K_1 \cap Y) \cap \cdots \cap (K_j \cap Y) = C_1 \cap \cdots \cap C_j = C_j
          \end{equation}
          which implies that $C_n = C_{n + 1} = \cdots$, so $Y$ is Noetherian.
          \item Recall that any quasi-compact subset of a quasi-compact Hausdorff space is closed. If $X$ is Noetherian, then every subset is Noetherian, thus quasi-compact, thus closed.
           Every subset being closed implies that $X$ has the discrete topology. Quasi-compactness of the whole space then implies that it is a finite set of points (if we had an infinite number of points,
           with each single-point set open, we could not have a finite subcover).
    \end{enumerate}
  \end{solution}

\begin{solution}[Problem 1.1.8]
  Before jumping into this solution, we must remark on something: when we decompose an affine algebraic set into its irreducible components, $Y = Y_1 \cup \cdots \cup Y_m$, each $Y_j$ is maximal
  in the sense that it is not contained in any strictly larger irreducible algebraic subset $Y' \subset Y$. Clearly, $Y' = \bigcup_{j} (Y' \cap Y_j)$, so if $Y'$ is irreducible, $Y' \cap Y_j = Y'$ for at least
  one of the $j$. It can't be $j = 2, \dots, m$ as we would then have $Y_1 \subset Y' \subset Y_j$. Thus, $Y' = Y_1$.
  This means that when given some affine algebraic set $V$ with irreducible decomposition $V = V(\mathfrak{p}_1) \cup \cdots \cup V(\mathfrak{p}_m)$ with each $\mathfrak{p}_j$ a prime ideal, then each of
  these primes is minimal in the sense that they contain no smaller prime which contains $I(V)$.

  We write $Y = V(I)$, where $I$ is prime, and $H = V(f)$, where $f$ is some irreducible polynomial, as well as $Y \cap H = V(I, f) = V(J)$ where $J = \text{Rad}(I, f)$.
  Let $V(\mathfrak{p}_j)$ be one of the irreducible components of $V(J)$, so $\mathfrak{p}_j \supset (I, f)$. We note that
  \begin{align}
    \dim V(\mathfrak{p}_j) = \dim k[x_1, \dots, x_n]/\mathfrak{p}_j &= \dim (k[x_1, \dots, x_n]/I)/(\mathfrak{p}_j/I)
    \\ &= \dim(k[x_1, \dots, x_n]/I) - \text{height}(\mathfrak{p}_j/I)
    \\ &= \dim V(I) - \text{height}(\mathfrak{p}_j/I)
    \\ &= r - \text{height}(\mathfrak{p}_j/I)
    \end{align}
  where $\mathfrak{p}_j/I$ is a prime ideal of $k[x_1, \dots, x_n]/I$. Suppose $\mathfrak{q}$ were another prime ideal such that $\mathfrak{q} \subset \mathfrak{p}_j/I$, then $I \subset \pi^{-1}(\mathfrak{q}) \subset \mathfrak{p}_j$,
  so by minimality of $\mathfrak{p}_j$, $\mathfrak{q} = \mathfrak{p}_j/I$. In addition, $f + I \in \mathfrak{p}_j/I$. We can't have $f \in I$, as then $V(f) \supset V(I)$,
  which we assumed is not the case. Clearly $f + I$ is not a zero-divisor as $k[x_1, \dots, x_n]/I$ is an integral domain, as $I$ is prime. Moreover, if $fg = 1 + I$, so $fg + i = 1$ for some $i \in I$ and $g \in k[x_1, \dots, x_n]$,
  then $(f, I) = (1)$, so $V(J) = \emptyset$. Thus, we actually require the addition assumption that $Y$ and $H$ intersect at all!

  With this new assumption, we can apply Krull's Haupidealsatz to see that $\text{height}(\mathfrak{p}_j/I) = 1$, so that $\dim V(\mathfrak{p}_j) = r - 1$ for all $j$.
  \end{solution}

\begin{solution}[Problem 1.1.9]
  We can do this problem inductively, assuming $\mathfrak{a}$ is proper. Obviously if $\mathfrak{a}$ is generated by a single, non-zero element $f$ which isn't a unit, since $k[x_1, \dots, x_n]$ is a UFD, we can
  factor $f = f_1 \cdots f_m$ where each $f_j$ is irreducible, so $V(f) = V(f_1) \cup \cdots \cup V(f_m)$. We then know from Proposition 1.13 that each $V(f_j)$ is a variety (thus irreducible) of dimension $n - 1$,
  so it follows that every irreducible component of $V(f)$ has dimension $n - 1$. If $\mathfrak{a} = (0)$, then $V(\mathfrak{a}) = \mathbb{A}^n(k)$, which is dimension $n$, so we have proved the base case.

  From here, suppose $\mathfrak{a} = (a_1, \dots, a_r)$. We then have
  \begin{equation}
    V(\mathfrak{a}) = V(a_1, \dots, a_r) = V(a_1, \dots, a_{r - 1}) \cap V(a_r) = \bigcup_{j} V(\mathfrak{p}_j) \cap V(a_r)
    \end{equation}
  where $V(\mathfrak{p}_j)$ are the irreducible components of $V(a_1, \dots, a_{r - 1})$, each of which have dimension greater than or equal to $n - r + 1$. If $V(\mathfrak{p}_j) \subset V(a_r)$, then
  $V(\mathfrak{p}_j) \cap V(a_r) = V(\mathfrak{p}_j)$. If $V(\mathfrak{p}_j)$ is not a subset of $V(a_r)$, then $a_r$ cannot be $0$. We have assumed $\mathfrak{a}$ is proper, so $a_r$ is also not a unit, and
  we can decompose it into a union of $V(f_i)$, where each $V(f_i)$ is a hypersurface of dimension $n - 1$. Then, from Problem 1.1.8 above, every irreducible component of $V(\mathfrak{p}_j) \cap V(f_i)$ has dimension
  greater than or equl to $n - r$. All together, we can write $V(\mathfrak{a})$ as a union of irreducible algebraic sets, all of dimension greater than or equal to $n - r$. This completes the proof.
  \end{solution}

\begin{solution}[Problem 1.1.10]
  This problem has multiple parts:
  \begin{enumerate}
    \item Of course, any chain $Z_0 \subset \cdots \subset Z_n$ of distinct, closed, irreducible subsets of $Y$ is a chain of distinct, closed, irreducible subsets of $X$, so $\dim(Y) \leq \dim(X)$.
    \item From the first part, $\dim U_i \leq \dim X$ for all $i$, so $\sup_i \dim U_i \leq \dim X$. Conversely, let $Z_0 \subset \cdots \subset Z_n$ be a chain of distinct, closed, irreducible subsets of $X$.
     For each $Z_{j - 1} \subset Z_{j}$, there must exist some $U_{i_j}$ such that $Z_{j - 1} \cap U_{i_j}$ is a proper subset of $Z_{j} \cap U_{i_j}$. Thus, if we let $U = U_{i_1} \cup \cdots \cup U_{i_n}$, then
     \item
     \item 
    \end{enumerate}
  \end{solution}

\begin{solution}[Problem 1.1.11]
  Our first claim is that $Y = V(x^4 - y^3, x^5 - z^3, y^5 - z^4)$. It is very clear that $Y \subset V(x^4 - y^3, x^5 - z^3, y^5 - z^4)$. To show the reverse inclusion, pick some $(x, y, z) \in V$.
  We can take roots, so pick some $t$ so that $t^3 = x$, so $y^3 = t^{12}$ and $z^3 = t^{15}$. As polynomials in $y$ and $z$, each has exactly three roots (as, again, we are working in an algebraically closed field).
  We denote cubic roots of unity as $j^0 = 1, j, j^2$, so that we must have $y = j^{\alpha} t^4 \in \{ t^4, j t^4, j^2 t^4\}$ and $z = j^{\beta} t^5 \in \{t^5, j t^5, j^2 t^5\}$. From here, we must also have
  $y^5 = z^4$, so $j^{5 \alpha - 4 \beta} = 1$. It follows that $5 \alpha - 4 \beta$ must be a multiple of $3$, for $\alpha, \beta \in \{0, 1, 2\}$. It is easy to verify that the only choice is $(\alpha, \beta) = (0, 0)$,
  so $y = t^4$ and $z = t^5$, which means that $(x, y, z) = (t^3, t^4, t^5)$.

  \pop{\textbf{TODO:} Finish this}
\end{solution}

\begin{solution}[Problem 1.1.12]
  Consider
  \begin{equation}
    \label{eq:1}
    (x^2 + i y^2 - 1)(x^2 - iy^2 - 1) = x^4 + y^4 - 2x^2 + 1
    \end{equation}
  in $\mathbb{R}[x, y]$, which vanishes only at points $(\pm 1, 0) \in \mathbb{A}^2(\mathbb{R})$. Each of these single points is itself a closed proper subset of the vanishing
  set of $p(x, y) = x^4 + y^4 - 2x^2 + 1$, so the vanishing set is not irreducible. To see that this polynomial is irreducible in $\mathbb{R}[x, y]$, suppose that it was reducible.
  Since any facotirzation of $p$ in $\mathbb{R}[x, y]$ would be a factorization in $\mathbb{C}[x, y]$, and $\mathbb{C}[x, y]$ is a UFD, it follows that one of the factors in Eq.~\eqref{eq:1}
  must be reducible. Each factor must be degree-$1$, so it follows that if a factor is reducible, it will vanish precisely on the union of two straight lines. However, note that $x^2 \pm i y^2 - 1$
  vanishes on the embedded circle $(\cos(\theta), (\mp i)^{1/2} \sin(\theta))$, where given any two lines, we can always find a point on the circle not contained in these lines.

  Thus, the decomposition in Eq.~\eqref{eq:1} is into irreducible factors, so $p$ is irreducible in $\mathbb{R}[x, y]$.
  \end{solution}

\subsection{Section 1.2}

\noindent \pop{Even after taking a course in algebraic curves, my relative comfort-level when working with projective and quasi-projective varieties is significantly lower than when working with
  their affine counterparts. Hopefully I'll fix this deficiency in the process of doing these problems.}

\begin{solution}[Problem 1.2.1]
  Let $\pi : \mathbb{A}^{n + 1}(k) - (0, \dots, 0) \rightarrow \mathbb{P}^n(k)$ be the quotient map. Recall that
  \begin{equation}
    V_p(\mathfrak{a}) = \{ y \in \mathbb{P}^{n} \ | \ \text{there exists} \ x \in \mathbb{A}^{n + 1} - (0, \dots, 0) \ \text{where} \ \pi(x) = y, f(x) = 0 \ \text{for all} \ f \in \mathfrak{a}^h\}
    \end{equation}
  where $\mathfrak{a}_h$ is the set of all homogeneous elements of $\mathfrak{a}$. It follows that if $x \in V_a(\mathfrak{a}) - (0, \dots, 0)$, then $\pi(x) \in V_p(\mathfrak{a})$. Similarly,
  if $x \in \pi^{-1}(V_p(\mathfrak{a}))$, then all homogeneous element of $\mathfrak{a}$ vanish at some $z \in \pi^{-1}(\pi(x))$, thus $x$ itself, and since $\mathfrak{a}$ is generated by homogeneous elements, it follows
  that all $f \in \mathfrak{a}$ vanish at $x$, so $x \in V_a(\mathfrak{a}) - (0, \dots, 0)$. We have therefore shown that
  \begin{equation}
    V_a(\mathfrak{a}) - (0, \dots, 0) = \pi^{-1}(V_p(\mathfrak{a}))
    \end{equation}
  It follows that if $f$ is homogeneous and vanishes on $V_p(\mathfrak{a})$, it vanishes on any homogeneous coordinates for any point in $V_p(\mathfrak{a})$, thus any point in $\pi^{-1}(V_p(\mathfrak{a})) = V_a(\mathfrak{a}) - (0, \dots, 0)$.
  If $\deg(f) > 0$, then $f$ in fact vanishes on $V_a(\mathfrak{a})$, and $f \in I_a(V_a(\mathfrak{a})) = \text{Rad}(\mathfrak{a})$. We have thus shown that $I_p(V_p(\mathfrak{a})) \cap S_{> 0} \subset \text{Rad}(\mathfrak{a})$. Note that
  any element of $S_0 = k$ clearly cannot be in $I_p(V_p(\mathfrak{a}))$ if $V_p(\mathfrak{a}) \neq \emptyset$, so in this case, $I_p(V_p(\mathfrak{a})) \subset \text{Rad}(\mathfrak{a})$.
  \end{solution}

\begin{solution}[Problem 1.2.2]
Of course, if $V_p(\mathfrak{a}) = \emptyset$, then $S \supset \text{Rad}(\mathfrak{a}) \supset I_p(V_p(\mathfrak{a})) \cap S_{> 0} = I_p(\emptyset) \cap S_{> 0} = S_{> 0}$.
It follows that if $\text{Rad}(\mathfrak{a})$ does not contain some $c \neq 0$ in $S_0 = k$, then it is $S_{> 0}$, and if it does, it is all of $S$. If $\text{Rad}(\mathfrak{a})$ is either $S$ or
$S_{> 0}$, then in either case, it contains each of the elements $x_0, \dots, x_{n}$. It follows that for each $j$ from $0$ to $n$, there is some $d_j$ where $x_j^{d_j} \in \mathfrak{a}$. Then if $D = d_0 \cdots d_n$,
each $x_j^{D}$ is in $\mathfrak{a}$. Let $M = (n + 1) D$, and note that any generator $x_0^{\alpha_0} \cdots x_n^{\alpha_n}$ will have some $\alpha_j \geq D$, so the generator is in $\mathfrak{a}$. Therefore, $S_M \subset \mathfrak{a}$.
If $S_M \subset \mathfrak{a}$ for some $M$, then $V_p(\mathfrak{a}) \subset V_p(S_M)$. If $[x_0, \dots, x_n] \in V_p(S_M)$, then $x_0^M = \cdots = x_n^M = 0$, which is not the case for any point of projective space, so $V_p(S_M) = \emptyset$
and $V_p(\mathfrak{a}) = \emptyset$ as well.
\end{solution}

\begin{solution}[Problem 1.2.3]
We go point-by-point:
\begin{enumerate}
  \item This is a trivial application of the definitions.
    \item Again, trivial.
      \item Clearly, if $f$ is a homogeneous polynomial which vanishes on $Y_1 \cup Y_2$, it vanishes on $Y_1$ and $Y_2$ individually. The
        converse is also true, this proves the claim.
        \item From Problem 1.2.1, we already know that if $V_p(\mathfrak{a}) \neq \emptyset$, then $I_p(V_p(\mathfrak{a})) \subset \text{Rad}(\mathfrak{a})$. To show the reverse inclusion, let us first show that $I_p(X)$: the ideal generated
          by all homogeneous polynomials vanishing on $X \subset \mathbb{P}^n$, is radical. Suppose $f^m \in I_p(X)$, where $f = f_0 + f_1 + \cdots + f_d$ is the decomposition into homogeneous components. Since $I_p(X)$ is a homogeneous ideal, $f_d^{m} \in I_p(X)$, so
          $f_d \in I_p(X)$, as it vanishes on $X$. It follows that $(f - f_d)^m \in I_p(X)$ (use the binomial expansion) and $f - f_d = f_0 + \cdots + f_{d - 1}$. Continue this process inductively to see that $f_j \in I_p(X)$ for each $j$, so $f \in I_p(X)$ as well.

          Clearly, if $f \in \mathfrak{a}$, with $f = f_0 + \cdots + f_d$, then each homogeneous component $f_j$ is also in $\mathfrak{a}$ (as it is a homogeneous ideal). Thus, obviously $f_j$ is homogeneous and vanishes on $V_p(\mathfrak{a})$, so $f_j \in I_p(V_p(\mathfrak{a}))$,
          so $f \in I_p(V_p(\mathfrak{a}))$ as well. Then, since $I_p(V_p(\mathfrak{a}))$ is radical, $\text{Rad}(\mathfrak{a}) \subset I_p(V_p(\mathfrak{a}))$, and we have the desired equality.
          \item Clearly $V_p(I_p(Y))$ is a closed set. Any homogeneous polynomial in $I_p(Y)$ will vanish on $Y$, so $Y \subset V_p(I_p(Y))$, thus implying that $\overline{Y} \subset V_p(I_p(Y))$. To show that this inclusion is an equality, suppose $Y \subset V_p(T)$,
            where $T$ is some collection of homogeneous polynomials. Any $f \in T$ vanishes on $Y$, so $f \in I_p(Y)^{h}$, the set of homogeneous elements in $I_p(Y)$. Thus, $T \subset I_p(Y)^{h}$, and
            \begin{equation}
              V_p(I_p(Y)) = V_p(I_p(Y)^{h}) \subset V_p(T)
              \end{equation}
            It follows that $\overline{Y} = V_p(I_p(Y))$, as desired.
  \end{enumerate}
  \end{solution}

\begin{solution}[Problem 1.2.4]
  Another multi-part question, another test of my will:
  \begin{enumerate}
    \item Note that if $V_p(\mathfrak{a})$ is an algebraic/closed subset of $\mathbb{P}^n$, in the case that $V_p(\mathfrak{a}) = \emptyset$, then $I_p(V_p(\mathfrak{a})) = S$ and $V_p(S) = \emptyset$.
      If $V_p(\mathfrak{a}) \neq \emptyset$, then $I_p(V_p(\mathfrak{a})) = \text{Rad}(\mathfrak{a})$ is a radical homogeneous ideal. It is not equal to $S_{+}$, as if this were the case, then by Problem 1.2.2, $V_p(\mathfrak{a}) = \emptyset$.
      We also have
      \begin{equation}
      V_p(I_p(V_p(\mathfrak{a}))) = \overline{V_p(\mathfrak{a})} = V_p(\mathfrak{a})
      \end{equation}
      Now, assume that $\mathfrak{a}$ is a radical homogeneous ideal not equal to $S_{+}$, then if $\mathfrak{a} = S$, $V_p(\mathfrak{a}) = \emptyset$, and $I_p(V_p(\mathfrak{a})) = I_p(\emptyset) = S$. Otheriwse, $V_p(\mathfrak{a})$ is a
      non-empty closed subset of $\mathbb{P}^n$, and $I_p(V_p(\mathfrak{a})) = \text{Rad}(\mathfrak{a}) = \mathfrak{a}$, as desired.
      \item If $Y \subset \mathbb{P}^n$ is irreducible, then given $fg \in I_p(Y)$, let us decompose into homogeneous components $f = f_0 + \cdots + f_r$ and $g = g_0 + \cdots + g_q$. Then $f_r g_q \in I_p(Y)$, so $Y \subset V_p(f_r g_q) = V_p(f_r) \cup V_p(g_q)$.
        We then have $Y = (Y \cap V_p(f_r)) \cup (Y \cap V_p(g_q))$, so either $Y \subset V_p(f_r)$ or $Y \subset V_p(g_q)$, so either $f_r \in I_p(Y)$ or $g_q \in I_p(Y)$. Then, either $(f - f_r) g \in I_p(Y)$ or $f (g - g_q) \in I_p(Y)$. We repeat
        this same argument inductively by looking at the top-degree homogeneous component of the new polynomial, until we eventually conclude that either $f_0, \dots, f_r \in I_p(Y)$ or $g_0, \dots, g_q \in I_p(Y)$. Thus, $f \in I_p(Y)$ or $g \in I_p(Y)$,
        and $I_p(Y)$ is therefore prime.
        On the other hand, suppose $Y = V_p(T_1) \cup V_p(T_2)$ where both closed sets are proper. Since there is some $x \in V_p(T_1)$ which is not in $V_p(T_2)$,
        there must exist some $f \in T_2$ which does not vanish at $x$, and is thus not in $I_p(V_p(T_1))$. We can choose a similar $g \in T_1$. Note that $fg$ is homogeneous
        and vanishes on $Y$, so is in $I_p(Y)$. However, neither $f$ nor $g$ is in $I_p(Y) = I_p(V_p(T_1)) \cap I_p(V_p(T_2))$, so $I_p(Y)$ is not prime.
        \item Note that $V_p(0) = \mathbb{P}^n$, where $(0)$ is prime homogeneous, so $I_p(V_p(0)) = \text{Rad}(0) = (0)$ is prime, so from above, $\mathbb{P}^n$ is irreducible.
    \end{enumerate}
  \end{solution}

\begin{solution}[Problem 1.2.5]
  Multiple parts. In both cases, the proofs are similar to their corresponding proofs in affine space:
  \begin{enumerate}
    \item Given a descending sequence of closed sets $V_p(\mathfrak{a}_1) \supset V_p(\mathfrak{a}_2) \supset \cdots$, we have an increasing sequence of homogeneous ideals $\mathfrak{a}_1 \subset \mathfrak{a}_2 \subset \cdots$
      in $k[x_0, \dots, x_n]$, a Noetherian ring, so this sequence eventually terminates, and thus the sequence of closed sets must as well.
      \item Consider the family $\mathcal{F}$ of all closed sets which cannot be written as a finite union of irreducible closed sets. Since $\mathbb{P}^n$ is Noetherian, this family has a minimal element $V$. $V$ cannot
        be irreducible itself, so $V = V_1 \cup V_2$, where $V_1$ and $V_2$ are proper closed subsets. It can't be the case that we can write both $V_1$ and $V_2$ as a finite union of irreducible closed subsets, as then
        we could write $V$ in this form, so either $V_1 \in \mathcal{F}$ or $V_2 \in \mathcal{F}$, which contradicts minimality.

        Suppose $V = V_1 \cup \cdots \cup V_n$ and $V = W_1 \cup \cdots \cup W_m$, where the $V_k$ and $W_k$ are irreducible closed, and no closed set contains another in each of the decompositions (obviously,
        such a decomposition exists). Note that $W_k = \cup_j (W_k \cap V_j)$, and since $W_k$
        is irreducible, and each $W_k \cap V_j$ is closed, we must have $W_k \cap V_j = W_k$ for some $j$, so $W_k \subset V_j$. Thus, each $W_k$ in the decomposition of contained in one of the
        $V_j$. We can use the same argument to show that each $V_j$ is contained in one of the $W_k$. In other words, given $W_k$, we have $V_j$ and $W_{\ell}$ such that $W_k \subset V_j \subset W_{\ell}$, and
        the no-subsets condition implies that $W_k = V_j = W_{\ell}$. This means that each of the $W_k$ is one of the $V_j$, so $W_1 \cup \cdots \cup W_m = V_{k_1} \cup \cdots \cup V_{k_m}$, with $W_j = V_{k_j}$,
        with each of the $V_{k_j}$ distinct. The fact that each $V_j$ is contained in one of the $W_{i} = V_{k_i}$ then implies that these decompositions are actually equal.
    \end{enumerate}
  \end{solution}

\begin{solution}[Problem 1.2.6]
This solution is quite involved, and will require a few preliminary results:
\begin{enumerate}
  \item The localization of an integral domain is an integral domain. This one is actually easy: if $\frac{a_1}{s_1} \frac{a_2}{s_2} = 0$ in $S^{-1} R$, then
    $s a_1 a_2 = 0$ in $R$ for some $s \in S$ with $s \neq 0$, so either $a_1 = 0$ or $a_2 = 0$.
    \item The prime ideals of $S^{-1} R$ are in bijective correspondence with the prime ideals of $R$ which do not intersect $S$. We won't prove this right now, maybe later in an appendix or something.
    \item If $R$ is an integral domain, then the fraction field of $R[t, t^{-1}]$ is isomorphic to $\text{Frac}(R)(t)$. We also won't prove this right now.
    \item Transcendence degree is additive, in the sense that if we have field extensions $L/K/k$, then
      \begin{equation}
        \text{tr} \deg_k(L) = \text{tr} \deg_K(L) + \text{tr} \deg_k(K)
        \end{equation}
      which implies that since $A(Y_i)[t, t^{-1}]$ is a finitely-generated $k$-algebra,
      \begin{align}
        \dim A(Y_i)[t, t^{-1}] &= \text{tr} \deg_k \text{Frac}(A(Y_i)[t, t^{-1}])
        \\ &= \text{tr} \deg_k \text{Frac}(A(Y_i))(t)
        \\ &= 1 + \text{tr} \deg_k \text{Frac}(A(Y_i)) = 1 + \dim(Y_i)
        \end{align}
      We won't prove this general, standard result about transcendence degree for now (maybe later).
      \item Localization commutes with quotients, in the sense that if $J$ is a prime ideal, $(S/J)^{-1}(R/J) \simeq S^{-1} R/S^{-1} J$. I will prove this one. First of all, note that if $J$
        is an ideal, then $S^{-1} J$ is an ideal in $S^{-1} R$. We can define $\phi((s + J)^{-1} (r + J)) = s^{-1} r + S^{-1} J$. To see that this is well defined, note that if $(s_1 + J)^{-1} (r_1 + J) = (s_2 + J)^{-1} (r_2 + J)$,
        then $s (s_1 r_2 - s_2 r_1) \in J$ for some $s \notin J$, which means that $s_1 r_2 - s_2 r_1 \in J$, so that $s_1^{-1} r_1 - s_2^{-1} r_2 \in S^{-1} J$. It is easy to check that this is a ring homomorphism.
        Surjectivity is obvious, and to see injectivity, note
        that if $s^{-1} r \in S^{-1} J$, then $s^{-1} r = t^{-1} q$ for $t \in S$ and $q \in J$, so there is some $s' \in S$ such that $s' t r = s' s q$, so $(s' t) r \in J$. Thus, by definition of the localization, $(s + J)^{-1} (r + J) = 0$,
        as we have identified an element of $S/J$ by which we can multiply $r + J$ to get $0$, namely $s' t + J$.

        This result will allow us to conclude that $S(Y)_{x_i + I_p(Y)} \simeq k[x_0, \dots, x_n]_{x_i}/I_p(Y)_{x_i}$, as $I_p(Y)$ is a prime ideal. Suppose $x_i^{M} \in I_p(Y)$ for some $M$, then $x_i \in I_p(Y)$, so $Y \subset V(x_i)$.
        Thus, if $Y$ is not a subset of $V(x_i)$, then $I_p(Y)_{x_i}$ is a prime ideal. Moreover, if $\mathfrak{p} \subset I_p(Y)$ is a prime ideal, then $\mathfrak{p}$ is obviously disjoint from all powers of $x_i$ in
        this case as well. It follows that there is a bijective correspondence between the prime ideals of $S(Y)$ contained in $I_p(Y)$ and the prime ideals contained in $I_p(Y)_{x_i}$. Thus,
        \begin{equation}
          \text{height}(I_p(Y)_{x_i}) = \text{height}(I_p(Y))
          \end{equation}
        so it follows that
        \begin{align}
          \dim S(Y)_{x_i + I_p(Y)} &= \dim k[x_0, \dots, x_i, x_i^{-1}, \dots, x_n] - \text{height}(I_p(Y))
          \\ &= \dim k[x_0, \dots, x_n] - \text{height}(I_p(Y))
          \\ &= \dim S(Y)
          \end{align}
        where we are using the pretty-easy-to-verify fact that $k[x_0, \dots, x_n]_{x_i} \simeq k[x_0, \dots, x_i, x_i^{-1}, \dots, x_n]$, which is a finitely-generated $k$ algebra and
        has the same fraction field as $k[x_0, \dots, x_n]$, thus the same dimension.
  \end{enumerate}

  Now, let us jump into the main body of the solution. Following the advice of the hint, let $\varphi_i : U_i \rightarrow \mathbb{A}^n$ be the homeomorphism from $U_i \subset \mathbb{P}^n$ to affine
  space given by sending $[x_0, \dots, x_n]$ to $(x_0/x_i, \dots, \widehat{x_i/x_i}, \dots, x_n/x_i)$. Suppose we choose $i$ so that $Y \cap U_i \neq \emptyset$.
  Consider the set $\varphi_i(Y \cap U_i)$, where $Y$ is closed and irreducible, so $Y \cap U_i$ is closed in $U_i$. It is also irreducible, as if $(C_1 \cap Y \cap U_i) \cup (C_2 \cap Y \cap U_i) = Y \cap U_i$, where $C_1$ and $C_2$ are closed in
  $\mathbb{P}^n$, then
  \begin{equation}
  Y = (Y \cap U_i) \cup (Y \cap U_i^{C}) = (C_1 \cap Y) \cup (C_2 \cap Y) \cup (Y \cap U_i^{C})
  \end{equation}
  Thus, either $Y \subset C_1$ or $Y \subset C_2$ or $Y \subset U_i^{C} = V(x_i)$, so $Y \cap U_i = C_1 \cap Y \cap U_i, C_2 \cap Y \cap U_i, \ \text{or} \ \emptyset$. We assumed $Y \cap U_i$ is non-empty, so we're done.
  It follows that $\varphi_i(Y \cap U_i)$ is closed and irreducible in $\mathbb{A}^n$, thus an affine variety.

  To be more specific, $Y = V_p(\mathfrak{a})$ for some homogeneous prime ideal $\mathfrak{a}$. Then
  \begin{align}
    \varphi_i(Y \cap U_i) &= \left\{ \left( \frac{x_0}{x_i}, \dots, \frac{x_n}{x_i} \right) \ \Big| \ x_i \neq 0, \ f(x_0, \dots, x_n) = 0 \ \text{for all} \ f \in \mathfrak{a} \right\}
    \\ &= \left\{ (y_1, \dots, y_n) \ | \ f_{*}(y_1, \dots, y_n) = 0 \ \text{for all} \ f \in \mathfrak{a} \right\}
    \\ &= \left\{ (y_1, \dots, y_n) \ | \ g(y_1, \dots, y_n) = 0 \ \text{for all} \ g \in \mathfrak{a}_{*} \right\}
    \end{align}
  where $f_{*}$ is the polynomial in $k[y_1, \dots, y_n]$ where we have substituted $x_j$ for $1$, and $\mathfrak{a}_{*}$ is the set of all $f_{*}$ for $f \in \mathfrak{a}$.
  It is easy to see that $\mathfrak{a}_{*}$ is an ideal, so the final line is irreducible algebraic set $V_a(\mathfrak{a}_{*})$.

  From here, let $Y_i = \varphi_i(Y \cap U_i) = V_a(\mathfrak{a}_{*})$, and let $A(Y_i)$ denote the coordinate ring. We construct a map $\phi : A(Y_i) \rightarrow S(Y)_{x_i + I_p(Y)}$ as follows.
  Given some $f + I(Y_i)$ in $A(Y_i)$, we write $f = f_0 + \cdots + f_d$ in terms of its homogeneous components. We then define
  \begin{equation}
    \phi(f(y_1, \dots, y_n) + I(Y_i)) = \sum_{k = 0}^{d} (x_i + I_p(Y))^{-k} (f_k(x_0, \dots, x_{i -  1}, x_{i + 1}, \dots, x_n) + I_p(Y))
    \end{equation}
  To prove that this map is well-defined, suppose $g \in I(Y_i) = \mathfrak{a}_{*}$, so there exists some $F \in \mathfrak{a}$ where $F_{*} = g$.
  Note that if $G$ is homogeneous, then $x_i^d (G_{*})^{*} = G$ for some $d$, where the upper-star is homogenization in the $i$-th variable. Moreover,
  \begin{align}
    (A + B)^{*} = \sum_{k = 0}^{\deg(A + B)} x^{\deg(A + B) - k}_i (A_k + B_k) &= \sum_{k = 0}^{\deg(A)} x^{\deg(A + B) - k}_i A_k + \sum_{k = 0}^{\deg(B)} x^{\deg(A + B) - k}_i B_k
    \\ &= x^{\deg(A + B) - \deg(A)}_i A^{*} + x^{\deg(A + B) - \deg(B)}_i B^{*}
    \end{align}
  which immediately means that
  \begin{equation}
  g^{*} = (F_*)^{*} = \left( \sum_{k = 0}^{d} (F_k)_{*} \right)^{*} = \sum_{k = 0}^{d} x_i^{d_k} ((F_k)_{*})^{*}
  \end{equation}
  It follows that is we multiply both sides by some $x_i^D$ for large enough $D$, we will have
  \begin{equation}
    x^D_i g^{*} = \sum_{k = 0}^{d} x_i^{D_k} F_k
    \end{equation}
  for some numbers $D_k$. Note that since $\mathfrak{a}$ is homogeneous, each $F_k$ is in $\mathfrak{a}$ (as $F \in \mathfrak{a}$), so the above sum is also in $\mathfrak{a}$, which means
  $x^D_i g^{*} \in \mathfrak{a} = I_p(Y)$. If we had $x_i^D \in \mathfrak{a}$, then $Y = V_p(\mathfrak{a}) \subset V_p(x_i)$, which we already assumed is not the case, so since $\mathfrak{a}$ is prime,
  $g^{*} \in \mathfrak{a}$. Note that
  \begin{equation}
    g^{*}(x_0, \dots, x_n) = \sum_{k = 0}^{d} x_i^{d - k} g_k(x_0, \dots, x_{i-1}, x_{i + 1}, \dots, x_n)
    \end{equation}
  which means that
  \begin{align}
    0 + I_p(Y) = g^{*} + I_p(Y) &= \sum_{k = 0}^{d} (x_i^{d - k} + I_p(Y)) (g_k + I_p(Y))
    \\ &= (x_i^{d} + I_p(Y)) \sum_{k = 0}^{d} (x_i + I_p(Y))^{-k} (g_k + I_p(Y))
    \end{align}
  in $S(Y)_{x_i + I_p(Y)}$. From here, we know that $x_i \notin \mathfrak{a}$, so since $S(Y)$ is an integral domain and the localization of an integral domain is an integral domain,
  it follows that $\sum_{k = 0}^{d} (x_i + I_p(Y))^{-k} (g_k + I_p(Y)) = 0$, as desired, and the map
  is well-defined. We can also verify that it is a homomorphism easily. Of course, the imgage of this homomorphism is in the collection of degree-$0$ element of
  $S(Y)_{x_i + I_p(Y)}$, and it is equally easy to see that this image is \emph{precisely} the set of these degree-$0$ elements (as it is easy to find a preimage). To verify injectivity, note that if
  \begin{equation}
    \sum_{k = 0}^{d} (x_i + I_p(Y))^{-k} (f_k + I_p(Y)) = (x_i + I_p(Y))^{-d} \sum_{k = 0}^{d} (x_i^{d - k} f_k + I_p(Y)) = 0
    \end{equation}
  then we must have $f^{*} = \sum_{k = 0}^{d} x_i^{d - k} f_k \in I_p(Y) = \mathfrak{a}$. This means that $(f^{*})_{*} \in \mathfrak{a}_{*}$, so $f = x_i^{d} (f^{*})_{*}$ is also in $\mathfrak{a} = I(Y_i)$.
  Thus, we have shown that $\phi$ is an injective homomorphism whose image is the set of all degree-$0$ elements of the localized ring $S(Y)_{x_i + I_p(Y)}$, and we have the desired identification.
  
  The next part of this problem asks us to show that the map we constructed above in fact extends to an isomorphism $S(Y)_{x_i + I_P(Y)} \simeq A(Y_i)[t, t^{-1}]$.
  To this end, we define
  \begin{align}
    \Phi((f + I(Y_i)) t^s) = \phi(f + I(Y_i)) (x_i^{s} + I_p(Y))
  \end{align}
  and extend linearly. It isn't difficult (but the notation sucks) to show that $\Phi$ is a ring homomorphism. It is also obvious that $\Phi$ is surjective (as, again, it is easy to find a preimage).
  To verify injectivity, suppose
  \begin{equation}
    \Phi \left( \sum_{s = 0}^{d} (f^{(s)} + I(Y_i)) \right) = \sum_{s = 0}^{d} \phi(f^{(s)} + I(Y_i)) (x_i^{s} + I_p(Y)) = 0
    \end{equation}
  Then, since each coefficient is degree-$0$, it follows that we must have $\phi(f^{(s)} + I(Y_i)) = 0$ for each $s$, so since $\phi$ is injective, $f^{(s)} + I(Y_i) = 0$ for
  each $s$, so we have injectivity, and therefore, $\Phi$ is a ring isomorphism.

  Now, since $Y \cap U_i \neq \emptyset$, so $Y$ is not a subset of $V(x_i)$, we have, using Point 4 and 5 from our list at the beginning of the solution, and the above isomorphism,
  \begin{equation}
    \dim S(Y) = \dim S(Y)_{x_i + I_p(Y)} = \dim A(Y_i)[t, t^{-1}] = \dim(Y_i) + 1
    \end{equation}
  Since $\varphi_i$ is a homeomorphism, it follows that $\dim(Y_i) = \dim(Y \cap U_i)$. Clearly, all of the sets $Y \cap U_i$ are an open cover for $Y$, so using
  Problem 1.1.10, we get $\dim(Y) = \max_{i} \dim(Y \cap U_i)$. It follows immediately that $\dim S(Y) = \dim(Y) + 1$. Moreover, since the $\dim S(Y) = \dim(Y_i) + 1$
  for all $i$ such that $Y \cap U_i \neq \emptyset$, it follows that in this case, we always have $\dim(Y_i) = \dim(Y)$.
  \end{solution}

\begin{solution}[Problem 1.2.7]
  Point-by-point:
  \begin{enumerate}
    \item We have already seen that $\mathbb{P}^n$ is a projective variety, $V_p(0)$, so $S(\mathbb{P}^n) = k[x_0, \dots, x_n]$. Thus,
      from the previous problem,
      \begin{equation}
        \dim(\mathbb{P}^n) = \dim k[x_0, \dots, x_n] - 1 = (n + 1) - 1 = n
        \end{equation}
      as desired.
      \end{enumerate}
  \end{solution}

\begin{solution}[Problem 1.2.8]

  \end{solution}

\subsection{Section 1.3}



\section{Chapter 2}

\subsection{Section 2.1}

\noindent \pop{Something that should be noted about Section 2.1 is that it isn't particularly conceptually deep, its main challenge is in getting acclimatized to some subtle definitions.
If one is comfortable with all of the definitions, there should be no issue solving all of the problems.}
\newline

\begin{solution}[Problem 2.1.1]
  Let $\mathcal{F}$ be a presheaf, recall that its sheafification $\mathcal{F}^{+}$ is obtained by setting $\mathcal{F}^{+}(U)$ to be all functions $s : U \rightarrow \sqcup_{p \in U} \mathcal{F}_p$
  such that $s(p) \in \mathcal{F}_p$ and for each $p \in U$, there is a neighbourhood $V$ of $p$ and $t \in \mathcal{F}(V)$ such that $s(q) = t_q$ for each $q \in V$ (where $t_q$ is the germ of $t$ at $q$).
  The arrows are simply restriction of functions. The morphism $\theta(U) : \mathcal{F}(U) \rightarrow \mathcal{F}^{+}(U)$ of Abelian groups is given by $\theta(U)(s)(p) = s_p$.

  If $\mathcal{F}$ is the constant presheaf associated to $A$ on $X$, note that $\mathcal{F}(U) = A$, so all of the stalks of $\mathcal{F}$ is isomorphic to $A$, and the Abelian group $\mathcal{F}^{+}(U)$
  is isomorphic to the group of functions $s : U \rightarrow A$ such that for each $p \in U$, there is a neighbourhood $V$ of $p$ and $t \in A$ such that $s(q) = t$ for each $q \in V$. In other words, $s : U \rightarrow A$
  is a locally constant function, which is the case if and only if it is continuous when $A$ is given the discrete topology. Hence, $\mathcal{F}^{+}(U)$ is isomorphic to $\mathcal{G}(U)$, where $\mathcal{G}$ is the
  constant sheaf associated to $A$ on $X$. We have such an isomorphism for all $U$, and the isomorphisms are compatible with the restriction maps, so we have an isomorphism of sheaves.
  \end{solution}

\begin{solution}[Problem 2.1.2]
  There are three parts to this question:
  \begin{enumerate}
    \item Once again, recall that the kernel presheaf of a presheaf morphism $\varphi : \mathcal{F} \rightarrow \mathcal{G}$ is given by $U \mapsto \text{Ker}(\varphi(U)) \subset \mathcal{F}(U)$. Of course, it follows that every section
      of $\text{Ker}(\varphi(U))$ is a section of $\mathcal{F}(U)$, so it follows that if $\varphi$ is a morphism of \emph{sheaves}, and $s \in \text{Ker}(\varphi(U))$ restricts to $0$ on a cover of $U$, then $s = 0$. Moreover, given
      a collection of sections $s_i \in \text{Ker}(\varphi(V_i))$ for a cover $\{V_i\}$ for $U$ which agree on overlaps, we can construct $s \in \mathcal{F}(U)$ which restricts to
      each of these sections, so $s|_{V_i} = s_i$. Then, we note that
      \begin{equation}
        \varphi(U)(s)|_{V_i} = \varphi(V_i)(s|_{V_i}) = \varphi(V_i)(s_i) = 0
        \end{equation}
      for each $V_i$, so since $\mathcal{G}$ is a sheaf, it follows that $\varphi(U)(s) = 0$, which means that $s \in \text{Ker}(\varphi(U))$. Thus, in this case, the kernel presheaf is a sheaf.

      From here, we need to show that $(\text{Ker}(\varphi))_p = \text{Ker}(\varphi_p)$ at each $p \in X$. Recall that the map $\varphi_p : \mathcal{F}_p \rightarrow \mathcal{G}_p$ of stalks
      is defined as $\varphi_p(U, s) = (U, \varphi(U)(s))$, so the kernel of $\varphi_p$ will consist precisely of germs $(U, s)$ around $p$ such that $\varphi(U)(s) = 0$ (i.e. $s \in \text{Ker}(\varphi(U))$).
      On the other hand, the stalk at $p$ of $\text{Ker}(\varphi)$ consists of germs $(U, s)$ around $p$ where $s \in \text{Ker}(\varphi)(U) = \text{Ker}(\varphi(U))$, so the two are obviously equal.

      Similarly, $\text{Im}(\varphi_p)$ is all germs $(U, \varphi(U)(s))$ around $p$ in $\mathcal{G}_p$. On the other hand, $\text{Im}(\varphi)$ is the sheafification of the presheaf $U \mapsto \text{Im}(\varphi(U))$, which
      we denote $\text{Im}(\varphi)^{\text{pre}}$.
      We know that stalks of the sheaf and the sheafification can be identified with each other. In particular, a germ of $\text{Im}(\varphi)_p$ is $(U, s)$, where $s : U \rightarrow \bigcup_{q \in U} \text{Im}(\varphi)^{\text{pre}}_q$
      is a function such that if $U$ is chosen to be small enough, then $s(q) = t_q$ for some section $t \in \text{Im}(\varphi(U)) \subset \mathcal{G}(U)$. It follows that we can define an isomorphism
      $\text{Im}(\varphi)_p \rightarrow \text{Im}(\varphi^{\text{pre}}_p$ which takes $(s, U) \mapsto (t, U)$.

      So, $\text{Im}(\varphi)_p$ is isomorphic collection of germs $(U, s)$ around $p$ where $s \in \text{Im}(\varphi(U))$, which is what we want.
      \item If $\varphi$ is injective, then by definition the sheaf $\text{Ker}(\varphi)$ is trivial, so $\text{Ker}(\varphi)_p = \text{Ker}(\varphi_p)$ is trivial for all $p$, so every homomorphism of
        stalks $\varphi_p$ is injective. On the other hand, if each $\varphi_p$ is trivial, then each stalk $\text{Ker}(\varphi)_p$ is trivial. It follows that if $s \in \text{Ker}(\varphi(U))$ is a section,
        then $s|_V = 0$ for some $V \subset U$ around $p$. We can cover $U$ by $V$ around each $p$ where $s|_V = 0$, so since $s$ is a section of a \emph{sheaf}, $s = 0$
        as well, and $\text{Ker}(\varphi)$ is trivial.

        Proving the similar result for surjectivity follows the same steps. In particular, we want to show that $\text{Im}(\varphi) = \mathcal{G}$ if and only if $\varphi_p : \mathcal{F}_p \rightarrow \mathcal{G}_p$ is surjective
        for all $p$. If we assume that the former holds, then $\text{Im}(\varphi_p) = \text{Im}(\varphi)_p = \mathcal{G}_p$ for all $p$ (these equalities are actually natural isomorphisms, but once again, we can
        permit ourselves to be a bit sloppy). On the other hand, if $\text{Im}(\varphi_p)$ and thus $\text{Im}(\varphi)_p$ is equal to $\mathcal{G}_p$ for all $p$, then we need to show that $\text{Im}(\varphi)(U) \simeq \mathcal{G}(U)$
        for all $U$ (or are at least naturally isomorphic). 
      \item 
    \end{enumerate}
  \end{solution}

\subsection{Section 2.2}

\begin{solution}[Problem 2.2.1]
  To produce such an isomorphism, we obviously need a homeomorphism between the underlying topological spaces
  $\phi : D(f) \rightarrow \text{Spec}(A_f)$ and an isomorphism of sheaves $\phi^{\#} : \mathcal{O} \rightarrow \phi_{*} \mathcal{O}_X|_{D(f)}$
  where $\mathcal{O}$ is the structure sheaf (such that the morphism induces a local homomorphism of stalks).

  If you look in Appendix.~\ref{appx:extra}, you will see that we have already constructed a bijection between $D(f)$ (the collection of all prime ideals which do not contain $(f)$, hence all prime
  ideals which do not intersect it) and $\text{Spec}(A_f)$
  (the prime ideals of the localization $A_f$). We denote this map by $\phi$ (i.e. $\phi : D(f) \rightarrow \text{Spec}(A_f)$ is given by $\phi(\mathfrak{p}) = S^{-1} \mathfrak{p}$ and $\phi^{-1}(\mathfrak{q}) = \rho^{-1}(\mathfrak{q})$,
  where $\rho : A \rightarrow S^{-1} A$ is the localization map with $S = \{1, f, f^2, \dots\}$). Our first claim is that $\phi^{-1}(V(\mathfrak{a})) = D(f) \cap V(\rho^{-1}(\mathfrak{a}))$. Indeed, note that if $\mathfrak{p}$ is a prime
  which contains $\mathfrak{a}$, then $\phi^{-1}(\mathfrak{p}) = \rho^{-1}(\mathfrak{p})$ is a prime which does not contain $f$ and which contains $\rho^{-1}(\mathfrak{a})$. Similarly, if $\mathfrak{p}$ is a prime containing
  $\rho^{-1}(\mathfrak{a})$ which does not contain $f$ then $\phi(\mathfrak{p})$ is prime and contains $(\phi \circ \rho^{-1})(\mathfrak{a}) = \mathfrak{a}$ (see Rem.~\ref{rem:s0}),
  so $\mathfrak{p} \in \phi^{-1}(V(\mathfrak{a}))$, and we have the desired equality. This implies that $\phi$ is continuous.

  On the other hand, we claim that $\phi(V(\mathfrak{a}) \cap D(f)) = V(\phi(\mathfrak{a}))$. If $\mathfrak{p}$ is prime, contains $\mathfrak{a}$, and does not contain $(f)$, then
  $\phi(\mathfrak{p})$ is prime and contains $\phi(\mathfrak{a})$. On the other hand, if $\mathfrak{p}$ is prime and contains $\phi(\mathfrak{a})$, then $\phi^{-1}(\mathfrak{p})$ is prime, doesn't contain $f$, and contains
  $(\rho^{-1} \circ \phi)(\mathfrak{a})$, which contains $\mathfrak{a}$, so $\mathfrak{p} \in \phi(V(\mathfrak{a}) \cap D(f))$. Thus, $\phi$ is in fact a homeomorphism.
  \newline

  \noindent We still need to produce an isomorphism of sheaves. We define $\phi^{\#}(U) : \mathcal{O}(U) \rightarrow \mathcal{O}_X(\phi^{-1}(U))$ as follows. Given a section
  $s : U \rightarrow \bigsqcup_{\phi(\mathfrak{p}) \in U} (A_f)_{\phi(\mathfrak{p})}$ in $\mathcal{O}(U)$, we can define function $\widetilde{s} : \phi^{-1}(U) \rightarrow \bigsqcup_{\mathfrak{p} \in \phi^{-1}(U)} A_{\mathfrak{p}}$
  as $\widetilde{s} = \Phi^{-1}_U \circ s \circ \phi|_{\varphi^{-1}(U)}$, where $\Phi_U : \bigsqcup_{\mathfrak{p} \in \phi^{-1}(U)} A_{\mathfrak{p}} \rightarrow \bigsqcup_{\phi(\mathfrak{p}) \in U} (A_f)_{\phi(\mathfrak{p})}$ is the unique map extending all of the isomorphisms
  $\Phi_{\mathfrak{p}} :  A_{\mathfrak{p}} \rightarrow (A_f)_{\phi(\mathfrak{p})}$ introduced in Cor.~\ref{cor:s3}, for all $\mathfrak{p} \in \phi^{-1}(U)$.
  This also gives us a candidate for $(\phi^{\#}(U))^{-1}$: the map taking $\widetilde{s}$ to $\Phi_U \circ s \circ (\phi|_{\varphi^{-1}(U)})^{-1}$.
  The fact that both of these maps are homomorphisms simply follows from the fact that we can add and multiply functions, and that $\Phi$ is a homomorphism when restricted to the individual rings. For example,
  \begin{equation}
    (\Phi^{-1} \circ s_1 \cdot s_2 \circ \phi)(\mathfrak{p}) = \Phi^{-1}(s_1(\phi(\mathfrak{p})) \cdot s_2(\phi(\mathfrak{p}))) = \Phi^{-1}(s_1(\phi(\mathfrak{p}))) \cdot \Phi^{-1}(s_2(\phi(\mathfrak{p})))
    \end{equation}
  To prove that this function is, in fact, an element of $\mathcal{O}_X(\phi^{-1}(U))$, first note that
  \begin{equation}
    \widetilde{s}(\mathfrak{p}) = (\Phi^{-1} \circ s \circ \phi)(\mathfrak{p}) = (\Phi^{-1} \circ s)(\phi(\mathfrak{p})) \in \Phi^{-1}((A_f)_{\phi(\mathfrak{p})}) = A_{\mathfrak{p}}
  \end{equation}
  so we satisfy the first condition. We also require that for each $\mathfrak{p} \in \phi^{-1}(U)$, there exists a neighbourhood $\mathfrak{p} \in V \subset \phi^{-1}(U)$ and $a, b \in A$ such
  that $\widetilde{s}(\mathfrak{q}) = \frac{a}{b} \in A_{\mathfrak{q}}$ for all $\mathfrak{q} \in V$ (in particular, $b \notin \mathfrak{q}$ for any $\mathfrak{q} \in V$).  It is of course the case
  that given some $\phi(\mathfrak{p}) \in U$, we can produce a neighbourhood $V$ such that $s = \frac{a'/f^n}{b'/f^m}$ on $V$. It then follows that on $\phi^{-1}(V) \subset \phi^{-1}(U)$, which
  contains $\mathfrak{p}$, we have
  \begin{equation}
    (\Phi^{-1} \circ s \circ \phi)(\mathfrak{q}) = \Phi^{-1} \left( \frac{a'/f^n}{b'/f^m} \right) = \frac{a' f^m}{b' f^n} \in A_{\mathfrak{q}}
  \end{equation}
  where $b' f^n \notin \mathfrak{q} \in \phi^{-1}(V)$, as if it were, then either $b' \in \mathfrak{q}$ or $f \in \mathfrak{q}$. The latter cannot happen by definition of $\mathfrak{q}$. If the former did happen, then
  we would have $\frac{b'}{f^m} \in \phi(\mathfrak{q}) \in V$, which is not the case. Thus, $\widetilde{s}$ is a valid element of the structure sheaf. Showing that the inverse homomorphism
  also outputs a valid element of the structure sheaf follows a similar procedure.
  \newline

  \noindent We now need to show that $\phi^{\#}$ is actually a natural transformation, which is to say that it behaves nicely with respect to restriction. Suppose $U$ and $V$ are open
  subsets of $\text{Spec}(A_f)$, with $V \subset U$. We need to show that $\phi^{\#}(s|_V) = \phi^{\#}(s)|_{\phi^{-1}(V)}$, with $s \in \mathcal{O}(U)$. This more or less follows from the
  definitions:
  \begin{align}
    \phi^{\#}(s|_V) = \Phi^{-1}_V \circ s|_V \circ \phi|_{\varphi^{-1}(V)} = (\Phi^{-1}_U \circ s \circ \phi)|_{\varphi^{-1}(V)} = \phi^{\#}(s)|_{\phi^{-1}(V)}
    \end{align}
  The same logic applies to the inverse mapping.
  \newline

  \noindent The last thing to show is that the induced homomorphisms $\phi^{\#}_{\phi(\mathfrak{p})} : \mathcal{O}_{\phi(\mathfrak{p})} \rightarrow (\phi_{*} \mathcal{O}_X|_{D(f)})_{\phi(\mathfrak{p})}$ on the stalks are local homomorphisms.
  A generic element of $\mathcal{O}_{\phi(\mathfrak{p})}$ is a germ $[U, s]$ where $U$ is open around $\phi(\mathfrak{p})$ and $s \in \mathcal{O}(U)$. A generic element of $(\phi_{*} \mathcal{O}_X|_{D(f)})_{\phi(\mathfrak{p})}$ is also
  of this form, with $s \in \mathcal{O}_X(\phi^{-1}(U))$, where $\phi^{-1}(U)$ is an open set around $\mathfrak{p}$ in $D(f)$. Because $\phi$ is a homeomorphism, it follows that this stalk is isomorphic to the stalk
  $(\mathcal{O}_X)_{\mathfrak{p}}$, via the map $[U, s] \mapsto [\varphi^{-1}(U), s]$. Via evaluation maps at $\phi(\mathfrak{p})$ and $\mathfrak{p}$, $\mathcal{O}_{\phi(\mathfrak{p})} \simeq (A_f)_{\phi(\mathfrak{p})}$ and
  $(\mathcal{O}_X)_{\mathfrak{p}} \simeq A_{\mathfrak{p}}$. Thus, when composing with these isomorphisms, the mapping of stalks takes an element $r$ of $A_{\mathfrak{p}}$, picks some germ corresponding to a
  section taking on this value at $\mathfrak{p}$, maps under $(\phi^{\#})^{-1}$, and evaluates at $\phi(\mathfrak{p})$, which all together, just yields $\Phi_{\mathfrak{p}}(r)$. Thus, the map of stalks is, up to conjugating by isomorphisms,
  the isomorphism $\Phi_{\mathfrak{p}}$ introduced in the appendix. Hence, we simply need to show that $\Phi_{\mathfrak{p}} : A_{\mathfrak{p}} \rightarrow (A_f)_{\phi(\mathfrak{p})}$ is a local map. To do this, note
  that the maximal ideal in $(A_f)_{\phi(\mathfrak{p})}$ is the ideal generated by $\phi(\mathfrak{p}) \subset A_f$, which is the ideal generated by all element of the form $\frac{p}{f^n}$ for $p \in \mathfrak{p}$. On the other hand,
  the maximal ideal of $\mathfrak{A}_{\mathfrak{p}}$ is the ideal generated by $\mathfrak{p}$. It is a pretty immediate consequence from the definition of $\Phi_{\mathfrak{p}}$, that the inverse image of the former
  maximal ideal is equal to the latter. Thus, we have a local homomorphism.
  \newline

  \noindent Thus, we have shown that as locally ringed spaces, $(D(f), O_X|_{D(f)}) \simeq \text{Spec}(A_f)$, as desired.
  \end{solution}

\begin{solution}[Problem 2.2.2]
  First consider the case where $X = \text{Spec}(A)$, and $\mathcal{O}_X$ is the structure sheaf, so it is an affine scheme. If $U \subset X$ is an open set, then $X - U = V(\mathfrak{a})$ for some
  ideal $\mathfrak{a} \subset A$, which has generating set $(f_{\alpha})_{\alpha \in J}$. It follows that $V(\mathfrak{a}) = \bigcap_{\alpha \in J} V(f_{\alpha})$, so $U = \bigcup_{\alpha \in J} D(f_{\alpha})$.
  In other words, we can cover $U$ with open sets of the form $D(f)$. We know that $(D(f), \mathcal{O}_X|_{D(f)})$ is an affine scheme from Problem 2.2.1, so it follows that we can cover $(U, \mathcal{O}_X|_U)$ in affine
  schemes, which implies it is a scheme (note that restricting sheaves to open sets behaves nicely, so restricting to $U$ and then restricting to $D(f)$ is the same as restricting to $D(f)$ from $\mathcal{O}_X$).

  To prove the general case, note that if $U$ is an open set of generic scheme $X$, then we can cover $X$ in open sets which are isomorphic to affine schemes, $(V_{\alpha}, \mathcal{O}_X|_{V_{\alpha}})$. It then
  follows that $(U \cap V_{\alpha}, \mathcal{O}_X|_{U \cap V_{\alpha}})$ is an open subset of an affine scheme, which means it is a scheme, which means we can cover it in affine schemes $(W_{\alpha, \beta}, \mathcal{O}_X|_{W_{\alpha, \beta}})$.
  Taking all of these affine schemes will cover $(U, \mathcal{O}_X|_U)$, which implies it is itself a scheme.
  \end{solution}

\begin{solution}[Problem 2.2.3]
  Point-by-point:
  \begin{enumerate}
    \item Suppose $s \in \mathcal{O}_X(U)$ is nilpontent, so that $s^N = 0$ for some $N$. Since restriction respects ring structure, $(s|_V)^N = (s^N)|_V = 0$ for some $V \subset U$,
      so the stalk around any $p \in U$ will have a nilpotent element, namely the germ $s_p$. On the other hand, suppose stalk $\mathcal{O}_{X, p}$ has nilpotent germ $s_p$, so
      $s_p^N = [s, U]^N = [s^N, U] = 0$ for some $N$. The $0$-germ $0$ at $p$ is simply $[0, V]$, for some open set $V$ around $p$, so we must have $s^N$ and $0$ agreeing on some neighbourhood
      $W \subset U \cap V$ of $p$. Therefore, $\mathcal{O}_X(W)$ contains a nilpotent element.
      \item Recall that the stalks of the sheafification $\mathcal{O}_{X, \text{red}}^{\#}$ are precisely the stalks of the presheaf $U \mapsto \mathcal{O}_X(U)_{\text{red}}$, which have no nilpotent elements.
        Therefore, if we prove that $(X, \mathcal{O}_{X, \text{red}}^{\#})$ is a scheme, it will be a reduced scheme.

        We will begin by assuming that $(X, \mathcal{O}_X) = (\text{Spec}(A), \mathcal{O})$.
        The idea is to show that $(\text{Spec}(A), (\mathcal{O}_{\text{red}}^{\#}) \simeq (\text{Spec}(A/\mathfrak{r}), \mathcal{O}')$, where $\mathfrak{r} = \text{Rad}(0)$, the nilradical of $A$. For any
        ideal $\mathfrak{a}$, we have the quotient map $\pi : A \rightarrow A/\mathfrak{a}$. This means that we have an induced map of locally ringed spaces
        $(f, f^{\#}) : (\text{Spec}(A/\mathfrak{a}), \mathcal{O}') \rightarrow (\text{Spec}(A), \mathcal{O})$, where $f(\mathfrak{p}) = \pi^{-1}(\mathfrak{p})$. If we restrict the image
        of $f$ to $V(\mathfrak{a})$, we actually find that $f : \text{Spec}(A/\mathfrak{a}) \rightarrow V(\mathfrak{a})$ is a homeomorphism. The bijective correspondence between the prime ideal of $A/\mathfrak{a}$ and the prime
        ideal of $A$ which contain $\mathfrak{a}$ is obvious. Continuity is obvious from the fact that $\pi$ is induced by the quotient homomorphism $A \rightarrow A/\mathfrak{a}$. It is also easy to see that
        for $\mathfrak{a} \subset \mathfrak{b}$, we have $\pi(V(\mathfrak{b})) = V(\pi(\mathfrak{b}))$. Therefore,
        \begin{equation}
        \text{Spec}(A/\mathfrak{r}) \simeq V(\mathfrak{r}) = V((0)) = \text{Spec}(A)
        \end{equation}
        as desired: $f : \text{Spec}(A/\mathfrak{r}) \rightarrow \text{Spec}(A)$ is a homeomorphism.

        The other thing we need to do is prove that the sheaves $\mathcal{O}_{\text{red}}^{\#}$ and $f_{*} \mathcal{O}'$ are isomorphic. Of course, $f^{\#} : \mathcal{O} \rightarrow f_{*} \mathcal{O}'$ is a local morphism
        of sheaves. Note that if $s \in \mathcal{O}(U)$, then $f^{\#}(U)(s) = \pi' \circ s \circ f = 0$ where $\pi'|_{A_{\mathfrak{p}}} : A_{\mathfrak{p}} \rightarrow (A/\mathfrak{r})_{\pi(\mathfrak{p})}$ is the reduction map,
        where we recall that localization commutes with quotients. If $s$ is nilpotent, then $s(\mathfrak{p}) \in \mathfrak{r}_{\mathfrak{p}}$ for every $\mathfrak{p} \in U$, so $f^{\#}(U)(s) = 0$. It follows that
        $f^{\#}(U) : \mathcal{O}(U) \rightarrow \mathcal{O}'(f^{-1}(U))$ descends to a morphism $f^{\#}_{\text{red}}(U) : \mathcal{O}(U)_{\text{red}} \rightarrow \mathcal{O}'(f^{-1}(U))$. Moreover, these morphisms
        clearly form a local morphism of presehaves. It follows that it extends a unique morphism of sheaves $g : \mathcal{O}_{\text{red}}^{\#} \rightarrow f_{*} \mathcal{O}'$. We want to show that
        this map is an isomorphism of sheaves, so we show that it induces an isomorphism on the stalks. Indeed, we have $\mathcal{O}_{\text{red}, f(p)}^{\#} = \mathcal{O}_{\text{red}, f(p)}$. $g_{f(p)}$ takes
        germs $[V, [s]]$ to germ $[V, \pi' \circ s \circ f]$, where $[s]$ is an equivalence class in $\mathcal{O}(V)_{\text{red}}$. If $\pi' \circ s \circ f = 0$ on $V$, then we must have $s(f(\mathfrak{p}))$ nilpotent
        for every $\mathfrak{p} \in f^{-1}(V)$, hence $s(\mathfrak{p})$ is nilpotent for every $\mathfrak{p} \in V$, so $[s] = 0$, and we have injectivity. Surjectivity follows from the fact that $\pi'$ is surjective and
        $f$ is a homeomorphism. Thus, we have our isomorphism of stalks, and the map as an isomorphism of sheaves.

        To prove the general case, we just apply the result above locally. In particular, if $(X, \mathcal{O}_X)$ is a scheme, then locally, $(U, \mathcal{O}_X|_U)$ is isomorphic to $(\text{Spec}(A), \mathcal{O})$. Thus,
        $(U, \mathcal{O}_{X}|_{U, \text{red}}^{\#})$ is locally isomorphic to $(\text{Spec}(A), \mathcal{O}_{\text{red}}^{\#})$, which is isomorphic to $(\text{Spec}(A/\mathfrak{r}), \mathcal{O}')$, which is itself
        an affine scheme.

        To prove the final thing, let $\text{red} : \mathcal{O}_X \rightarrow \mathcal{O}_{X, \text{red}}^{\#}$ be the unique morphism of schemes which extends the morphism of preschemes
        $\text{red} : \mathcal{O}_X \rightarrow \mathcal{O}_{X, \text{red}}$ which takes $s \in \mathcal{O}_X(U)$ to its reduction $[s] \in \mathcal{O}_X(U)_{\text{red}}$. The morphism
        $(\text{id}, \text{red}) : X_{\text{red}} \rightarrow X$ is a morphism of schemes which is a homeomorphism on the underlying topological spaces.

        \item Of course, if we write $f = (h, h^{\#})$, then $g$ will be the mapping $h$ on the underlying topological spaces. We require the existence of unique $g^{\#} : \mathcal{O}_{Y, \text{red}}^{\#} \rightarrow h_{*} \mathcal{O}_X$
          such that $g^{\#} \circ \text{red} = h^{\#}$. If we can prove the existence of a unique $\chi : \mathcal{O}_{Y, \text{red}} \rightarrow h_{*} \mathcal{O}_X$ (morphism of presheaves) where $\chi \circ \text{red} = h^{\#}$ as
          presheaf morphisms, where $\text{red}$ in this context is the presheaf morphism taking a section to its reduction, then by uniqueness of extension to the sheafification, $g^{\#}$ is the unique extension of $\chi$.
          The idea is to simply define $\chi(V) : \mathcal{O}_Y(V)_{\text{red}} \rightarrow \mathcal{O}_X(h^{-1}(V))$ as $\chi(V)([s]) = h^{\#}(s)$. This is well-defined because if $s \in \mathcal{O}_Y(V)$ is nilpontent, then
          $h^{\#}(s)$ must be nilpotent in $\mathcal{O}_X(h^{-1}(V))$, which means it must be equal to $0$ as $X$ is reduced. It follows that if $s_1$ and $s_2$ differ by a nilpotent element, then $h^{\#}(s_1) = h^{\#}(s_2)$.
          It is easy to see that this is a local morphism of presheaves from here, and is unique via the formula we have used to define it. This completes the proof.
    \end{enumerate}
  \end{solution}

\begin{solution}[Problem 2.2.4]
  We want to show that $\alpha : \text{Hom}_{\text{Sch}}(X, \text{Spec}(A)) \rightarrow \text{Hom}_{\text{Ring}}(A, \Gamma(X, \mathcal{O}_X))$ is a bijection. In the case that $X = \text{Spec}(R)$ is an \emph{affine} scheme,
  this is quite simple. We replace $\Gamma(X, \mathcal{O}_X)$ with $R$, via the isomorphism, in the definition of $\alpha$ above.
  Note tha we have the map $\beta : \text{Hom}_{\text{Ring}}(A, R) \rightarrow \text{Hom}_{\text{Sch}}(\text{Spec}(R), \text{Spec}(A))$ which takes $\phi : A \rightarrow R$ to the induced map (Proposition 2.3). It is proved in
  Proposition 2.3 that $\beta$ is a left-inverse of $\alpha$. To prove that it is also a right inverse, note that $\phi$ induces $(f, f^{\#})$ where $f(\mathfrak{p}) = \phi^{-1}(\mathfrak{p})$, and we have
  the map $f^{\#}(\text{Spec}(A)) : \Gamma(\text{Spec}(A), \mathcal{O}_{\text{Spec}(A)}) \rightarrow \Gamma(\text{Spec}(R), \mathcal{O}_{\text{Spec}(R)})$ which takes $s_a$, the constant
  section taking on value $a \in A$ at every point, to $\phi \circ s_a \circ f$, which takes on $\phi(a)$ at every point. Thus, this section can be identified with $\phi$ under the isomorphisms identifying the
  global sections with $A$ and $R$ respectively. It follows that $\beta$ is a right inverse, and $\alpha$ is a bijection in this case.

  To prove that this result holds for general schemes, pick some cover $\{U_{\alpha}\}$ for $X$ such that $X$ is affine when restricted to each open set. We can use the $\alpha$ maps to induce a bijection
  \begin{equation}
    \widetilde{\alpha} : \displaystyle\prod_{\alpha \in J} \text{Hom}_{\text{Sch}}(U_{\alpha}, \text{Spec}(A)) \rightarrow \displaystyle\prod_{\alpha \in J} \text{Hom}_{\text{Ring}}(A, \Gamma(U_{\alpha}, \mathcal{O}_X))
    \end{equation}
  Of course, we have the natural restriction map on each affine open set, yielding
  \begin{equation}
  R : \text{Hom}_{\text{Sch}}(X, \text{Spec}(A)) \rightarrow  \displaystyle\prod_{\alpha \in J} \text{Hom}_{\text{Sch}}(U_{\alpha}, \text{Spec}(A))
  \end{equation}
  On the other hand, suppose we have a collection of morphisms of schemes $(f_{\alpha}, f_{\alpha}^{\#}) : U_{\alpha} \rightarrow \text{Spec}(A)$ such that $f_{\alpha} = f_{\beta}$ on $U_{\alpha} \cap U_{\beta}$ and
  given $f_{\alpha}^{\#}(V) : \mathcal{O}(V) \rightarrow \mathcal{O}_X(f_{\alpha}^{-1}(V))$ and $f_{\beta}^{\#}(V) : \mathcal{O}(V) \rightarrow \mathcal{O}_X(f_{\beta}^{-1}(V))$, we have
  \begin{equation}
    f_{\alpha}^{\#}(V)(s)|_{W} = f_{\beta}^{\#}(V)(s)|_{W}
    \end{equation}
  for $W \subset f_{\alpha}^{-1}(V) \cap f_{\beta}^{-1}(V)$. We immediately can combine the $f_{\alpha}$ to form a continuous map $f : X \rightarrow \text{Spec}(A)$. Moreover, given some open $V \subset \text{Spec}(A)$, and
  section $s \in \mathcal{O}(V)$, note that $f^{-1}(V)$ is covered by the $U_{\alpha}$ and $U_{\alpha} \cap f^{-1}(V) = f_{\alpha}^{-1}(V)$, so we can write $f^{-1}(V)$ as the union of the $f_{\alpha}^{-1}(V)$. Of course,
  the goal is to define a section on all of $f^{-1}(V)$. We have sections $f_{\alpha}^{\#}(V)(s) \in \mathcal{O}_X(f_{\alpha}^{-1}(V))$ for each $\alpha$, and we know that these sections agree on intersections.
  Thus, by definition of a sheaf, they glue together to a unique section which restricts to the subsections.

  It is easy to check that the subset of $\prod_{\alpha} \text{Hom}_{\text{Sch}}(U_{\alpha}, \text{Spec}(A))$ which is the image of $R$ is exactly the set with the restrictions described above, and that the map
  described above is an inverse to $R$. We use similar logic to treat the natural map
  \begin{equation}
    Q : \text{Hom}_{\text{Ring}}(A, \Gamma(X, \mathcal{O}_X)) \rightarrow \displaystyle\prod_{\alpha \in J} \text{Hom}_{\text{Ring}}(A, \Gamma(U_{\alpha}, \mathcal{O}_X))
    \end{equation}
  where we restrict global sections to local sections. In particular, given some element of the product determined by a collection of homomorphisms $\phi_{\alpha}$, we consider the case
  where sections $\phi_{\alpha}(a)$ and $\phi_{\beta}(a)$ agree on $U_{\alpha} \cap U_{\beta}$. It is clear that these elements are the image of $Q$. Moreover, when restricting to the image of
  $R$, we can easily see that these elements are the image of $\widetilde{\alpha}|_{\text{Im}(R)}$. In particular, the maps $f_{\alpha}^{\#}(\text{Spec}(A))$ and $f_{\beta}^{\#}(\text{Spec}(A))$
  which take $\Gamma(\text{Spec}(A), \mathcal{O})$ to $\Gamma(U_{\alpha}, \mathcal{O}_X)$ and $\Gamma(U_{\beta}, \mathcal{O}_X)$ respectively take global sections of $\text{Spec}(A)$ to pairs
  which agree on $U_{\alpha} \cap U_{\beta}$.

  Moreover, given some collection of $\phi_{\alpha}$ which satisfy the desired criteria, they will induce an element of $\text{Im}(R)$.

  Thus, we have a commutative square where three of the arrows are bijections, so the fourth arrow between $\text{Hom}_{\text{Sch}}(X, \text{Spec}(A))$ and $\text{Hom}_{\text{Ring}}(A, \Gamma(X, \mathcal{O}_X))$ is also a
  bijection.
  \end{solution}

\begin{solution}[Problem 2.2.5]
  This follows quite readily from the previous solution: we have a bijection $\alpha : \text{Hom}_{\text{Sch}}(X, \text{Spec}(A)) \rightarrow \text{Hom}_{\text{Ring}}(A, \Gamma(X, \mathcal{O}_X))$. Thus,
  our bijection is between $\text{Hom}_{\text{Sch}}(X, \text{Spec}(\mathbb{Z}))$ and $\text{Hom}_{\text{Ring}}(\mathbb{Z}, \Gamma(X, \mathcal{O}_X))$. But of course, if $\phi : \mathbb{Z} \rightarrow R$ is an
  arbitrary ring homomorphism, then $\phi(1) = 1_R$, and this determines where every element of the domain is sent. Therefore, there is a unique element in $\text{Hom}_{\text{Sch}}(X, \text{Spec}(\mathbb{Z}))$, as desired.

  We are also asked to describe $\text{Spec}(\mathbb{Z})$: it is the collection of prime ideals $(p)$, each of which are closed points. The closed sets $V(k)$ can be broken down via a prime factorization of $k$
  to get a finite union of the single point sets $V(p_1) \cup \cdots \cup V(p_n)$, so along with the entire set $\text{Spec}(\mathbb{Z})$, these finite sets give the topology (this is usually called the finite-complement topology).

  \end{solution}

\begin{solution}[Problem 2.2.6]

  \end{solution}

\begin{solution}[Problem 2.2.7]
  A morphism is of the form $(f, f^{\#}) : \text{Spec}(K) \rightarrow X$. We know that $\text{Spec}(K)$ consists of a single point, $(0)$, so specifying $f$ is equivalent to specifying a point $x = f((0)) \in X$.
  The morphism of sheaves $f^{\#} : \mathcal{O}_X \rightarrow f_{*} \mathcal{O}_{\text{Spec}(K)}$ is equivalent to specifying local ring morphisms $f^{\#}(U) : \mathcal{O}_X(U) \rightarrow \mathcal{O}_{\text{Spec}(K)}((0))$
  for all $U$ which contain the point $x$. Of course, the ring $\mathcal{O}_{\text{Spec}(K)}((0))$ consists of all functions $s : \{(0)\} \rightarrow K_{(0)} \simeq K$, which is isomorphic to $K$ itself. Hence,
  we may specify a local ring morphism $\phi(U) : \mathcal{O}_X(U) \rightarrow K$ for each $U$ containing $x$, such that if $V \subset U$ and $\iota$ is the inclusion, then $\phi(U) = \phi(V) \circ \mathcal{O}_X(\iota)$.
  Specifying all such morphisms is clearly equivalent to specifying a local ring morphism $\phi : \mathcal{O}_{x} \rightarrow K$. Finally, if $\phi$ is a local ring morphism, then $\phi^{-1}(0) = \mathfrak{m}_x$, so
  we obtain an injective morphism $\widetilde{\phi} : \mathcal{O}_x/\mathfrak{m}_x \rightarrow K$. Conversely, if $\widetilde{\phi}$ is a pre-specified injective morphism, then $\phi = \widetilde{\phi} \circ \pi$,
  where we pre-compose with the projection map, is a local ring homomorphism $\mathcal{O}_x \rightarrow K$, and it is easy to see that these correspondences are inverses of each other.
  \end{solution}

\begin{solution}[Problem 2.2.8]
  Note that we have the natural inclusion $k \rightarrow k[\varepsilon]/\varepsilon^2$, which yields natural morphism $\text{Spec} k[\varepsilon]/\varepsilon^2 \rightarrow \text{Spec}(k)$, making
  $\text{Spec} k[\varepsilon]/\varepsilon^2$ a scheme over $k$, with map $\text{Spec}(\iota)$. A $k$-morphism of this scheme to $f : X \rightarrow \text{Spec}(k)$ is a morphism of schemes $g$ such that $f \circ g = \text{Spec}(\iota)$.
  Let $\mathfrak{p}$ be a prime ideal in $k[\varepsilon]/\varepsilon^2$, it is easy to see that the only possible non-zero
  proper ideals are principal, $([\varepsilon + a])$. Moreover, we must have $a = 0$, as $\varepsilon^2 + a \varepsilon$ is
  in $[\varepsilon + a]$, which means $a\varepsilon$ is, so $a^2$ is. It is clear that $([\varepsilon])$ is prime. Thus, $\text{Spec} k[\varepsilon]/\varepsilon^2$ consists of one point, $(\varepsilon)$
  ($(0)$ is not prime, as $\varepsilon^2 = 0$). \pop{finish}
  \end{solution}

\begin{solution}[Problem 2.2.9]

  \end{solution}

\begin{solution}[Problem 2.2.10]
  It is a basic fact from algebra that any (monic) polynomial in $\mathbb{R}[x]$ can be factored as a product of monomials $x - a$ and irreducible quadratics $x^2 + bx + c$. Thus, Since $\mathbb{R}$ is
  Noetherian, so is $\mathbb{R}[x]$. If $\mathfrak{p} = (f_1, \dots, f_n)$ is a prime ideal, then it must contain each of the irreducible factors of each $f_j$ of the above form, so any prime is generated
  by a finite collection of irreducibles of the form introduced in the first sentence. We cannot have $x - a$ and $x - b$ in this generating set for $a \neq b$, as then $1 \in \mathfrak{p}$ by taking their difference
  to obtain a unit. In addition, if $\mathfrak{p}$ contained both $x - a$ and $x^2 + bx + c$, then $\mathfrak{p}$ contains
  \begin{equation}
    x^2 + bx + c - (x - a)(x + b + a) = c + ab + a^2
    \end{equation}
  This must be equal to $0$, otherwise $\mathfrak{p}$ would contain a unit, but then $a$ would be a root of $x^2 + bx + c$, which can't be as this polynomial is irreducible. Thus, the prime ideals are precisely the
  principal ones $(x - a)$ and $(x^2 + bx + c)$ generated by the irreducible elements. Therefore, $\text{Spec}(\mathbb{R}[x])$ has a point corresponnding to each $a \in \mathbb{R}$, namely $(x - a)$. It also
  has a point corresponding to each \emph{conjugate pair} of complex numbers in $\mathbb{C}$. Thus, $\text{Spec}(\mathbb{R}[x])$ has more points than $\mathbb{R}$, but less than $\mathbb{C}$.

  If we want to compare topologies, note that of $\mathfrak{a} = (g_1, \dots, g_m)$ is a polynomial ideal, we have $V(\mathfrak{a}) = \cap_{j = 1}^{m} V(g_j)$. Each $V(g_j)$ is the union of $V(g_j^{(k)})$, where
  the $g_j^{(k)}$ are the irreducible factors. Each of these is a one-point set, so $V(g_j)$ is a finite union of points, and $V(\mathfrak{a})$ is also a finite union of points. Thus, the closed sets are precisely
  the finite collection of points, and the topology is thus the finite complement topology: quite different from the usual topologies on $\mathbb{R}$ and $\mathbb{C}$!
  \end{solution}

\begin{solution}[Problem 2.2.16]
  Point-by-point:

  \begin{enumerate}
    \item Let us first look at the affine case. In particular, when $X = \text{Spec}(B)$, a choice of a global section amount to a choice of element $f$ in $B$, the stalk $\mathcal{O}_{\mathfrak{p}}$ at point $\mathfrak{p}$
      is isomorphic to $A_{\mathfrak{p}}$, and $\mathfrak{m}_{\mathfrak{p}} \simeq \phi(\mathfrak{p})$, the image of $\mathfrak{p}$ in the localization. The stalk of a section is then just the image of $f$ in each
      of the localized rings $A_{\mathfrak{p}}$. The condition of $\rho(f)$ not being in $\phi(\mathfrak{p})$ is equivalent to the condition that $f \notin \rho^{-1}(\phi(\mathfrak{p})) = \mathfrak{p}$ (see the appendix).
      It follows that $\mathfrak{p} \in X_f$ if and only if $\mathfrak{p}$ does not contain $(f)$ (equivalent to $\mathfrak{p} \in D((f))$). So, $X_f = D((f))$, in the affine case.

      Now, for the general case, we can restrict to affine $U$ and restrict a section $f$ to $\overline{f}$, and note from above that $X_f \cap U = U_f = D((f))$.
      \item Again, it is easy to prove this in the affine case. In this case, $X_f = D((f))$. In this case, restricting some $a \in A$, thought of as a global section, to $\mathcal{O}(D(f)) \simeq A_f$ amounts to
        mapping it under inclusion into the localization. $a$ is $0$ in the localization if and only if $f^n a = 0$ in $A$, for some $n$. To prove the general case, take a finite affine open cover for $X$
        by $U_1, \dots, U_m$, and gets some $n_k$ for each $U_k$ so that $f^{n_k} a = 0$ in $\Gamma(U_k, \mathcal{O}_X|_{U_k})$. Taking the largest of the $n_k$ to be $n$, and we get $f^n a = 0$ on all of the open
        sets $U_k$, thus on all of $X$.
        \item Let us first look at the affines $U_i$. In particular, let us look at $b$ restricted to $X_f \cap U_i \simeq D(f)$, so $b \in \mathcal{O}(D(f)) \simeq A_f$. It follows that $b = \frac{a_i}{f^{n_i}}$
          for $a_i \in A$, on $U_i \cap X_f$. On the intersections $U_i \cap U_j \cap X_f = (U_i \cap U_j)_f$, we have course must have
          \begin{equation}
            \frac{a_i}{f^{n_i}} = \frac{a_j}{f^{n_j}} \Longleftrightarrow f^{m_{ij}} (a_i f^{n_j} - a_j f^{n_i}) = 0
            \end{equation}
          Then, since these intersections are quasi-compact, it follows from above that there exists some $M_{ij}$ such that $f^{M_{ij}} (a_i f^{n_j} - a_j f^{n_i}) = 0$
          on all of $U_i \cap U_j$. Let $M$ be the largest of all the $M_{ij}$. We then have $f^{M} (a_i f^{n_j} - a_j f^{n_i})$ for every pair $i$ and $j$. It follows that we just define
          $a = a_i f^{M + n_1 + \cdots + \hat{n_i} + \cdots + n_K}$ for every $i$. Let $N = n_1 + \cdots + n_K$. Then, when we restrict to $U_i \cap X_f$, we have $a = f^{M + N} b$, for each $i$,
          and we are done.
          \item Given some $b \in \Gamma(X_f, \mathcal{O}_{X_f})$, we have that $f^n b$ is the restriction of some $a \in A$. Thus, we map $b$ to $\frac{a}{f^n}$ in $A_f$.
            This is well-defined because if $f^m b$ is also a restriction of some $a' \in A$, we have
            \begin{equation}
              f^m a = f^{n + m} b = f^n a' \Longrightarrow \frac{a}{f^n} = \frac{a'}{f^m} \in A_f
              \end{equation}
            It is clear that this map is also a ring homomorphism. If $f^n b$ is the restriction of $0$, so $f^n b = 0$, then locally, where $b$ can be thought of as an element of $\mathcal{O}(D(f)) \simeq A_f$,
            it follows that $b = 0$, so we have injectivity. Surjectivity follows from the fact that given some $\frac{a}{f^n}$, we can always choose $b$ to be of this form locally in $\mathcal{O}(D(f))$, such a $b$ will be mapped
            to $\frac{a}{f^n}$. Thus, we have the desired isomorphism.
    \end{enumerate}
  \end{solution}

\begin{solution}[Problem 2.2.18]
Comparing properties of ring homomorphisms to the induce morphisms of schemes. Let's go!
\begin{enumerate}
  \item If $D(f)$ is empty if and only if every prime ideal of $A$ contains $(f)$ if and only if $f$ is in the intersection of all prime ideal of $A$, $\text{Rad}(0)$, if and only if $f^n = 0$ for some $n$.
    \item 
  \end{enumerate}
  \end{solution}

\subsection{Section 2.3}

\section{Chapter 3}

\appendix

\section{Extra proofs}
\label{appx:extra}

\begin{remark}[Some details on localizations]
  \label{rem:s0}
  Let $R$ be a ring, let $S$ be a multiplicative set. Let $\rho : R \rightarrow S^{-1} R$ be the localization map taking $r$ to $\frac{r}{1} \in S^{-1} R$. If $\mathfrak{a}$ is a (prime) ideal in $S^{-1} R$, then
  $\rho^{-1}(\mathfrak{a})$ is a (prime) ideal in $R$. We also define a map $\phi$ taking an ideal $\mathfrak{a}$ of $R$ to the ideal in $S^{-1} R$ generated by the set $\rho(\mathfrak{a})$. We will use the notation
  $\phi(\mathfrak{a}) = S^{-1} \mathfrak{a}$. It is easy to see that every element of $S^{-1} \mathfrak{a}$ is of the form $\frac{r}{s}$, for some $r \in \mathfrak{a}$. Let us consider the relationship
  between $\phi$ and the map $\rho^{-1}$ taking $\mathfrak{a}$ to $\rho^{-1}(\mathfrak{a})$. Our first claim is that $\mathfrak{a} = \phi(\rho^{-1}(\mathfrak{a}))$ for \emph{any} ideal $\mathfrak{a}$. To see this,
  pick some $\frac{r}{s} \in \mathfrak{a} \subset S^{-1} R$. Note that $\rho(r) = \frac{r}{1} = \frac{s r}{s} \in \mathfrak{a}$, so $r \in \rho^{-1}(\mathfrak{a}) \subset R$ and $\frac{r}{1} \in \rho(\rho^{-1}(\mathfrak{a}))$.
  This means that $\frac{r}{s}$ is
  in the ideal in $S^{-1} R$ generated by all images of elements of $\rho^{-1}(\mathfrak{a})$ under $\rho$, so $\mathfrak{a} \subset \phi(\rho^{-1}(\mathfrak{a}))$. On the other hand, given some element of $\phi(\rho^{-1}(\mathfrak{a}))$,
  it will be of the form
  \begin{equation}
    \frac{r_1}{s_1} \cdot \frac{q_1}{1} + \cdots + \frac{r_m}{s_m} \cdot \frac{q_m}{1} = \frac{r_1' q_1 + \cdots + r_m' q_m}{s_1 \cdots s_m}
    \end{equation}
  so any such element is of the form $\frac{r}{s}$ for some $r \in \rho^{-1}(\mathfrak{a})$. Thus, $\frac{r}{1} \in \mathfrak{a}$, so $\frac{r}{s} \in \mathfrak{a}$, which gives the reverse inclusion.

  Thus, $(\phi \circ \rho^{-1})(\mathfrak{a}) = \mathfrak{a}$. However, it is \emph{not} the case that $(\rho^{-1} \circ \phi)(\mathfrak{a}) = \mathfrak{a}$. This only works if we restrict the $\phi$ to the collection
  of prime ideals which do not intersect $S$, and $\rho^{-1}$ to the collection of prime ideals in $S^{-1} R$. Note that if $\mathfrak{p} \subset S^{-1} R$ is prime, then $\rho^{-1}(\mathfrak{p})$ will be prime
  in $R$. Moreover, if $s \in \rho^{-1}(\mathfrak{p})$ for some $s \in S$, then $\frac{s}{1} \in \mathfrak{p}$, so $\frac{1}{1} \in \mathfrak{p}$, contradicting the fact that it must be proper. On the other hand,
  if $\mathfrak{p}$ is a prime ideal in $R$ not intersecting $S$, then every element of $\phi(\mathfrak{p})$ is of the form
  \begin{equation}
    \frac{r_1}{s_1} \cdot \frac{p_1}{1} + \cdots + \frac{r_m}{s_m} \cdot \frac{p_m}{1} = \frac{p}{s}
    \end{equation}
  for $p \in \mathfrak{p}$, $s \in S$. Thus, if some $\frac{p}{s} \in \phi(\mathfrak{p})$ with $p \in \mathfrak{p}$, $s \in S$,
  can be written as $\frac{p}{s} = \frac{p_1}{s_1} \cdot \frac{p_2}{s_2} = \frac{p_1 p_2}{s_1 s_2}$, then $q s_1 s_2 p =  qs p_1 p_2$ for some $q \in S$, so $qs p_1 p_2 \in \mathfrak{p}$. Since $S \cap \mathfrak{p} = \emptyset$
  and $\mathfrak{p}$ is prime, $p_1 p_2 \in \mathfrak{p}$, so either $p_1$ or $p_2$ is. Then either $\frac{p_1}{s_1}$ or $\frac{p_2}{s_2}$ is in $\phi(\mathfrak{p})$.

  Let us finally show that $(\rho^{-1} \circ \phi)(\mathfrak{p}) = \mathfrak{p}$ for such prime ideals. First note that given $p \in \mathfrak{p}$, then $\rho(p) = \frac{p}{1}$ is in $\phi(\mathfrak{p})$, so
  $\mathfrak{p} \subset (\rho^{-1} \circ \phi)(\mathfrak{p})$. Note that we haven't yet used the fact that $\mathfrak{p}$ is prime: this inclusion holds for any ideal. However, we need this assumption for the reverse inclusion.
  If $p \in (\rho^{-1} \circ \phi)(\mathfrak{p})$, then $\frac{p}{1} \in \phi(\mathfrak{p})$. Thus, it can be written as $\frac{q}{s}$ for some $q \in \mathfrak{p}$ and $s \in S$. Then we have
  $t (sp - q) = 0$ for some $t \in S$, so $sp \in \mathfrak{p}$. Then, since $\mathfrak{p}$ is prime and does not contain any elements of $S$, $p \in \mathfrak{p}$, giving the reverse inclusion.
  \end{remark}

\noindent The above discussion gives us the following result:

\begin{lemma}
  \label{lem:s}
  If $R$ is a ring, and $S$ is a multiplicative set, then the prime ideals of $S^{-1} R$ are in bijective correspondence with the prime ideals of $R$ which do not intersect $S$
  \end{lemma}

\noindent Let us continue our discussion of localizations.

\begin{lemma}
  \label{lem:s2}
  If $S$ and $T$ are multiplicative sets in ring $R$ such that $S \subset T$, then $S^{-1} T$ is a multiplicative set and $T^{-1} R \simeq (S^{-1} T)^{-1} (S^{-1} R)$ as rings.
  \end{lemma}

\begin{proof}
Define the map $\Phi : T^{-1} R \rightarrow (S^{-1} T)^{-1} (S^{-1} R)$ to be $\Phi(\frac{r}{t}) = \frac{r/1}{t/1}$. To show that this is well-defined, suppose $\frac{r'}{t'} = \frac{r}{t}$. Then
$q (t r' - t' r) = 0$ for some $q \in T$. We want to show that $\frac{r'/1}{t'/1} = \frac{r/1}{t/1}$, which is true if and only if there is $\frac{x}{s} \in S^{-1} T$ such that

\begin{equation}
  \frac{x}{s} \left( \frac{t' r}{1} - \frac{t r'}{1} \right) = \frac{x (t' r - t r')}{s} = 0
  \end{equation}
so we just let $\frac{x}{s} = \frac{q}{1}$, and we have what we want. The fact that this is a homomorphism is easy. Given some $\frac{r/s_1}{t/s_2}$, note that $\Phi(\frac{s_2 r}{s_1 t}) = \frac{s_2 r / 1}{s_1 t / 1}$.
We claim this is equal to the first expression. Indeed,
\begin{equation}
  \frac{s_1 t}{1} \frac{r}{s_1} - \frac{s_2 r}{1} \frac{t}{s_2} = \frac{r}{t} - \frac{r}{t} = 0
  \end{equation}
so this is the case, and we have surjectivity. Finally, suppose $\Phi(\frac{r}{t}) = \frac{r/1}{t/1} = 0$. Then $\frac{x}{s} \frac{r}{1} = \frac{xr}{s} = 0$ for some $\frac{x}{s} \in S^{-1} T$. Then $x' r = 0$ for some $x' \in T$.
It follows that $\frac{r}{t} = \frac{x' r}{x' t} = 0$ in $T^{-1} R$, so we have injectivity. Thus, $\Phi$ is an isomorphism.
  \end{proof}

\begin{corollary}
  \label{cor:s3}
  It follows that if $A$ is a ring, and $\mathfrak{p}$ is a prime ideal which does not contain $f$, so $S = \{1, f, f^2, \dots\}$ is contained in $T = A - \mathfrak{p}$, then
  \begin{equation}
    A_{\mathfrak{p}} = T^{-1} A \simeq (S^{-1} T)^{-1} (S^{-1} A) = (S^{-1} (A - \mathfrak{p}))^{-1} (S^{-1} A) = (A_f - S^{-1} \mathfrak{p})^{-1} A_f = (A_f)_{S^{-1} \mathfrak{p}}
    \end{equation}
  This will be useful in some of the problems.
  \end{corollary}

\noindent Here are few facts about sheaves:

\begin{prop}
  If $f : X \rightarrow Y$ is a homeomorphism and $\mathcal{O}_X$ is a sheaf on $X$, then $(f_{*} \mathcal{O}_X)_{f(x)}$, the stalk of the pushforward sheaf, is
  isomorphic to $\mathcal{O}_{X, x}$.
  \end{prop}

\begin{proof}
  Note that the elements of $\mathcal{O}_{X, x}$ are germs $[U, s]$ where $s \in \mathcal{O}_X(U)$ and $x \in U$. The elements of $(f_{*} \mathcal{O}_X)_{f(x)}$ are germs $[V, r]$
  with $r \in \mathcal{O}_X(f^{-1}(V))$ and $f(x) \in V$. Thus, we have map $f^{\#} : \mathcal{O}_{X, x} \rightarrow (f_{*} \mathcal{O}_X)_{f(x)}$ with $f^{\#}[U, s] = [f(U), s]$.
  This is obviously a ring homomorphism, and it has an inverse.
  \end{proof}

\noindent I want to fill in some of the details in Hartshorne's proof of Proposition 2.3.

\begin{claim}
  If $\varphi : A \rightarrow B$ is a homomorphism of rings, then it induces a natural morphism of locally ringed spaces $(f, f^{\#}) : (\text{Spec}(B), \mathcal{O}_{\text{Spec}(B)}) \rightarrow (\text{Spec}(A), \mathcal{O}_{\text{Spec}(A)})$.
  \end{claim}

\begin{proof}
  We can define $f : \text{Spec}(B) \rightarrow \text{Spec}(A)$ as $f(\mathfrak{p}) = \varphi^{-1}(\mathfrak{p})$. This is clearly a continuous map, as
  \begin{equation}
    f^{-1}(V(\mathfrak{a})) = \{ \mathfrak{p} \ | \ \varphi^{-1}(\mathfrak{p}) \supset \mathfrak{a}\} = \{ \mathfrak{p} \ | \ \mathfrak{p} \supset \varphi(\mathfrak{a})\}
    \end{equation}
  and $\mathfrak{p}$ contains $\varphi(\mathfrak{a})$ if and only if it contains the ideal generated by $\varphi(\mathfrak{a})$, so this is just $V((\varphi(\mathfrak{a})))$.
  To define $f^{\#} : \mathcal{O}_{\text{Spec}(A)} \rightarrow f_{*} \mathcal{O}_{\text{Spec}(B)}$, note that
  we can define ``localized'' morphisms $\varphi_{\mathfrak{p}} : A_{f(\mathfrak{p})} \rightarrow B_{\mathfrak{p}}$ for prime $\mathfrak{p} \subset B$. We can then define
  $\varphi_U : \bigsqcup_{f(\mathfrak{p}) \in f(U)} A_{f(\mathfrak{p})} \rightarrow \bigsqcup_{\mathfrak{p} \in U} B_{\mathfrak{p}}$ by extending the $\varphi_{\mathfrak{p}}$
  is the obvious way. This allows us to define
  \begin{equation}
    f^{\#}(U)(s) = \varphi_U \circ s \circ f
    \end{equation}
  This collection of maps clearly behaves will with respect to restrictions, so it is a morphism of sheaves. The last thing we need to show is that the morphisms on stalks
  are local morphisms. We know that $\mathcal{O}_{\text{Spec}(A), f(\mathfrak{p})} \simeq A_{f(\mathfrak{p})}$, via the evaluation map (same with $\mathcal{O}_{\text{Spec}(B), \mathfrak{p}} \simeq B_{\mathfrak{p}}$.
  Therefore, the map $f^{\#}_{\mathfrak{p}} : \mathcal{O}_{\text{Spec}(A), f(\mathfrak{p})} \rightarrow \mathcal{O}_{\text{Spec}(B), \mathfrak{p}}$ immediately reduces to $\varphi_{\mathfrak{p}}$, which is a local
  homomorphism as it is the localization of a ring morphism.
  \end{proof}

\begin{claim}
  $X = \text{Spec}(A)$ is irreducible if and only if $\text{Rad}(0) \subset A$ is prime.
  \end{claim}
\begin{proof}
  Clearly, if $\mathfrak{p}$ is prime, then if
  $V(\mathfrak{p}) = V(\mathfrak{a}) \cup V(\mathfrak{b}) = V(\mathfrak{a} \mathfrak{b})$, we must have
  $\mathfrak{a} \mathfrak{b} \subset \mathfrak{p}$. One of the ideals must be in $\mathfrak{p}$, so we have
  $V(\mathfrak{p}) = V(\mathfrak{a})$ or $V(\mathfrak{b})$, which means it is irreducible. Conversely, if $V(\mathfrak{p})$
  is irreducible, suppose there is some $ab \in \mathfrak{p}$ where neither $a$ nor $b$ is in $\mathfrak{p}$. Then we have
  $(\mathfrak{p} + a) \cdot (\mathfrak{p} + b) = \mathfrak{p}$, so $V(\mathfrak{p}) = V(\mathfrak{p} + a) \cup V(\mathfrak{p} + b)$, a contradiction. It follows that $X = \text{Spec}(A) = V(\text{Rad}(0))$ is irreducible if and only if $\text{Rad}(0)$
  is prime.
  \end{proof}

\begin{claim}
$X$ is reduced if and only if $\text{Rad}(0) = 0$.
  \end{claim}
\begin{proof}
Obviously, if all the stalks $\mathcal{O}_{\mathfrak{p}} \simeq A_{\mathfrak{p}}$ contain no nilpotent elements, then
neither will $A$, so $\text{Rad}(0) = 0$. On the other hand, if $\text{Rad}(0) = 0$, suppose some $s^{-1} a \in A_{\mathfrak{p}}$ were nilpotent, so $\frac{a^n}{s^n} = 0$. Then there is $q \in A - \mathfrak{p}$ such that $q a^n = 0$. Then, $(qa)^n = 0$,
so $qa \in \text{Rad}(0)$, so $qa = 0$. Then $\frac{a}{s} = \frac{qa}{qs} = 0$.
  \end{proof}

\begin{claim}
  $X$ is integral if and only if $A$ is an integral domain.
  \end{claim}
\begin{proof}
  Obviously, if each $\mathcal{O}_X(U)$ is an integral domain, then $\mathcal{O}_X(X) \simeq A$ is. Conversely, if $A$
  is an integral domain, suppose $\mathfrak{p}$ is a prime ideal, and look at $A_{\mathfrak{p}}$. Suppose we have $\frac{a}{s}$ and $\frac{b}{r}$ such that $\frac{ab}{rs} = 0$. Then there exists $q \in A$ (with $q \neq 0$) with $q a b = 0$. Thus,
  $a = 0$ or $b = 0$, as $A$ is an integral domain. It follows that if $a, b \in \mathcal{O}_X(U)$ are sections
  where $ab = 0$, then at each point, we have $a(\mathfrak{p}) b(\mathfrak{p}) = 0$ implying $a(\mathfrak{p}) = 0$
  or $b(\mathfrak{p}) = 0$. We know that given $\mathfrak{p} \in U$, we will then have (WLOG) $a(\mathfrak{q}) = \frac{p}{q} = 0$ for $\mathfrak{q} \in V$ around $\mathfrak{p}$. In particular,
  $q \notin \mathfrak{q}$ in $V$. It follows that for each $\mathfrak{q} \in V$, we must have $s(\mathfrak{q}) p = 0$ for some non-zero $s(\mathfrak{q}) \in A - \mathfrak{q}$. Except, since $A$ is an integral domain,
  then $p = 0$. Therefore, $a(\mathfrak{p}) = 0$ for all $\mathfrak{p}$ in some open subset of $U$. Since $A$ is an integral domain, $(0)$ is prime, and is contained in every open set. Thus, any other open set on which
  $a$ is constant must be $0$, as the intersection of this open set with $V$ is non-trivial. Hence, $a = 0$, and $\mathcal{O}_X(U)$ is an integral domain.
  \end{proof}

\section{A few solutions (to more basic problems) from Vakil's book}

\begin{solution}[Problem 1.2.G]
  Define $\phi : \mathbb{Z}/(10) \otimes \mathbb{Z}/(12) \rightarrow \mathbb{Z}/(2)$ as $\phi([m] \otimes [n]) = [mn]$. This is well-defined because $mn \sim (m + 10a) (n + 12b)$ in $\mathbb{Z}/(2)$.
  It is clearly a well-defined homomorphism, and is clearly surjective. To check injectivity, note that if $[mn] = 0$, then either $m$ or $n$ is even. If $m = 2k$, and by Bezout we know that there
  exist $x$ and $y$ with $10x + 12y = (10, 12) = 2$, then
  \begin{equation}
    [m] \otimes [n] = [2k] \otimes [n] = ([10kx] \otimes [n]) + ([ky] \otimes [12n]) = 0
    \end{equation}
  Similar logic applies when $n$ is even, so $\phi$ is an isomorphism.
  \end{solution}

\section{Commutative algebra stuff}

\end{document}
