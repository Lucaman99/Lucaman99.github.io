\documentclass[aps,pra,showpacs,notitlepage,onecolumn,superscriptaddress,nofootinbib,oneside,10pt]{article}
\usepackage[utf8]{inputenc}
\usepackage[tmargin=1in, bmargin=1.25in, lmargin=1.6in, rmargin=1.6in]{geometry}
\usepackage{amsmath, amssymb, amsthm}
\usepackage{graphicx}
\usepackage{xcolor}
\usepackage{enumitem}
\usepackage{datetime}
\usepackage{hyperref}
\usepackage{titlesec}
\usepackage{import}
\usepackage{mathtools}
\usepackage{thmtools,thm-restate}
\usepackage{tikz-cd}
\usepackage[many]{tcolorbox}

% package for commutative diagrams
% \usepackage{tikz-cd}

%%%%%%%%%%%%%%%%%%%%%%%%%%%%%%%%%%%%%%%%%%%%%
\definecolor{crimson}{RGB}{186,0,44}
\definecolor{moss}{RGB}{0, 186, 111}
\newcommand{\pop}[1]{\textcolor{crimson}{#1}}
\newcommand{\zcom}[1]{\noindent\textcolor{crimson}{(Z): #1}}
\newcommand{\jcom}[1]{\noindent\textcolor{moss}{(J): #1}}
\newcommand{\wt}[1]{\widetilde{#1}}
\newcommand{\pqeq}{\succcurlyeq}
\newcommand{\pleq}{\preccurlyeq}
\newcommand{\mf}[1]{\mathfrak{#1}}

%%%%%%%%%%%%%%%%%%%%%%%%%%%%%%%%%%%%%%%%%%%%%
\hypersetup{
    colorlinks,
    linkcolor={crimson},
    citecolor={crimson},
    urlcolor={crimson}
}

\usepackage{qcircuit}
\usepackage{comment}

%%%%%%%%%%%%%%%%%%%%%%%%%%%%%%%%%%%%%%%%%%%%%
\theoremstyle{definition}
\newtheorem{definition}{Definition}[section]

\newtheorem{lemma}{Lemma}[section]

\newtheorem{theorem}{Theorem}[section]

\newtheorem{corollary}{Corollary}[theorem]
\newtheorem*{theorem*}{Theorem}
\newtheorem*{corollary*}{Corollary}

\newtheorem{remark}{Remark}[section]

\newtheorem{conjecture}{Conjecture}[section]
\newtheorem{example}{Example}[section]
\newtheorem{reminder}{Reminder}[section]
\newtheorem{problem}{Problem}[section]
\newtheorem{question}{Question}[section]
\newtheorem{answer}{Answer}[section]
\newtheorem{fact}{Fact}[section]
\newtheorem{claim}{Claim}[section]
\newtheorem{prop}{Proposition}[section]
\newtheorem{man}{Mantra}[section]

\newtheorem{solution}{Solution}[section]

\usepackage{geometry}
\geometry{
  left=25mm,
  right=25mm,
  top=20mm,
}

\newcommand{\hhrulefill}{\hspace{-1.5em} \hrulefill}
\renewcommand{\baselinestretch}{1.05}

%%%%%%%%%%%%%%%%%%%%%%%%%%%%%%%%%%%%%%%%%%%%%
\bibliographystyle{unsrt}

%%%%%%%%%%%%%%%%%%%%%%%%%%%%%%%%%%%%%%%%%%%%%
%%%%%%%%%%%%%%%%%%%%%%%%%%%%%%%%%%%%%%%%%%%%%
%%%%%%%%%%%%%%%%%%%%%%%%%%%%%%%%%%%%%%%%%%%%%
\begin{document}

\title{Chapter 1 solutions: \emph{An introduction to homological algebra} by Weibel}
\author{Jack Ceroni}
\date{\today}
\maketitle

\tableofcontents

\section{Section 1.2}

\begin{solution}[Exercise 1.2.5]
  Recall that in an Abelian category $\mathcal{A}$,
  \begin{equation}
    \text{Tot}(C)_n = \text{Tot}^{\Pi}(C)_n = \prod_{p + q = n} C_{p, q}
    \end{equation}
  which becomes a chain complex in $\mathcal{A}$ when endowed with differential $d = d^h + d^v$ (we are using the anti-commutative convention here).
  We must show that $H_n(\text{Tot}(C)) = 0$ for all $n$, equivalently, that $d$ along with the total complex $\text{Tot}(C)$ is exact, so in other words $\text{Im}(d_{n+1}) = \text{Ker}(d_{n})$.
  Let $i : K \rightarrow \text{Tot}(C)_n$ and $j : M \rightarrow \text{Tot}(C)_n$ be the inclusions of the kernel and image respectively. We are going to show that
  \begin{equation}
    \text{Ker}(d_n) = \text{Im}(d_n) + \text{Im}(d^v_{p, q + 1})
    \end{equation}
  for any $p + q = n$ via induction. Note that since our complex is bounded, $C_{p, q}$ will eventually be $0$ along the diagonal defined by $n$, so eventually, $\text{Im}(d^v_{p, q + 1}) = 0$,
  which will give the result. So, we first note that $d_n \circ i = d^v_n \circ i + d^h_n \circ i = 0$ (we are implicitly projecting from the product here). This means that $d^v_n \circ i, d^h_n \circ i = 0$. In other words, 
\end{solution}

\bibliography{refs}

\end{document}
